\documentclass[a4paper,12pt]{article}
\usepackage[left=2.5cm,right=2.5cm,top=2.5cm,bottom=2.5cm]{geometry} % Adjust page margins
\usepackage{xcolor,graphicx,framed}
\usepackage[normalem]{ulem}
\usepackage{amsmath}
\usepackage{cases}
\usepackage{gensymb}
\usepackage{chemmacros}
\setlength{\extrarowheight}{0.4cm}

\begin{document}

\newcommand{\HRule}{\rule{\linewidth}{0.4mm}} % Defines a new command for the horizontal lines, change thickness here

%----------------------------------------------------------------------------------------
%	HEADING SECTIONS
%----------------------------------------------------------------------------------------

\begin{minipage}{0.7\textwidth}
\begin{flushleft} 
\textsc{Universidad del Valle de Guatemala \\
Campus Central \\
Facultad de Ciencias y Humanidades \\
Departamento de Qu\'imica \\
Segundo ciclo, 2014 \\
Fisicoqu\'imica 1 \\
Secci\'on 10 \\
}
\end{flushleft}
\end{minipage}
~
\begin{minipage}{0.2\textwidth}
\begin{flushright}
\includegraphics[scale=0.3]{Logo_UVG} % Include a department/university logo
\end{flushright}
\end{minipage}\\

%----------------------------------------------------------------------------------------
%	TITLE SECTION
%----------------------------------------------------------------------------------------

\begin{center}
\HRule \\[0.4cm]
{ \bfseries Evaluaci\'on 3}\\ % Title of your document
\HRule \\[0.4cm]
\end{center}

%----------------------------------------------------------------------------------------

\section*{Instrucciones}

Responder y resolver los siguientes problemas en las hojas adicionales haciendo uso de calculadora cient\'ifica, una tabla peri\'odica y el formulario individual que haya sido previamente aprobado para su uso. Cualquier actitud deshonesta ser\'a sancionada seg\'un el C\'odigo de comportamiento de los estudiantes de la Universidad.

\section*{Problemas}

\begin{enumerate}

 \item (10 pts) Un estudiante busc\'o los valores de $\Delta_f\bar{G}^\standardstate$, $\Delta_f\bar{H}^\standardstate$ y $\bar{S}^\standardstate$ del $\mbox{CO}_2$ en el ap\'endice de un libro de Fisicoqu\'imica. Al usar estos valores, encontr\'o que: 
$$\Delta_f\bar{G}^\standardstate\neq \Delta_f\bar{H}^\standardstate-T\bar{S}^\standardstate$$ 
a $298\;\mbox{K}$. Indicar qu\'e reacci\'on est\'a involucrada en algunos de esos valores y por qu\'e es incorrecto lo que intent\'o realizar el estudiante. % Problema 6.14 de Chang

 \item (30 pts) Esbozar el diagrama de fase del alcohol terbut\'ilico (o 2-metil-2-propanol, con f\'ormula: $\mbox{(H}_3\mbox{C)}_3\mbox{COH}$) usando la siguiente informaci\'on: punto triple, $298.96\;\mbox{K}$ y $0.053\;\mbox{bar}$; punto cr\'itico, $506.2\;\mbox{K}$ y $39.7\;\mbox{bar}$; punto de fusi\'on normal, $298.3\;\mbox{K}$; punto de ebullici\'on normal, $355.5\;\mbox{K}$. Indicar las regiones en que las distintas fases son m\'as estables y tomar en cuenta la forma t\'eorica de las curvas. % Problema sobre diagrama de fase

 \item (20 pts) Un monta\~nista escala el Monte Everest y prepara una taza de t\'e. Pone a ebullir al agua en su jarilla, pero al final su t\'e sabe muy mal. Dado que la presi\'on en la cima de la monta\~na es $0.4\times P^\standardstate$, calcular la temperatura a la cu\'al ebull\'o el agua. La entalp\'ia de vaporizaci\'on del agua es $50\;\mbox{kJ}\cdot\mbox{mol}^{-1}$ y el punto de ebullici\'on est\'andar del agua es $99.97\celsius$. % Problema de Monk

 \item (20 pts) Cuando cierto l\'iquido de masa molar $46.1\;\mbox{g}\cdot\mbox{mol}^{-1}$ se congela a $-3.65\celsius$, su densidad cambia de $0.789\;\mbox{g}\cdot\mbox{cm}^{-3}$ a $0.801\;\mbox{g}\cdot\mbox{cm}^{-3}$. Su entalp\'ia de fusi\'on es $8.68\;\mbox{kJ}\cdot\mbox{mol}^{-1}$. Estimar el punto de congelaci\'on del l\'iquido a $100\;\mbox{MPa}$. % Problema 4.12(b) de Atkins

 \item (20 pts) A $20\celsius$, la densidad de una soluci\'on de etanol/agua al 20 por cierto en peso es $968.7\;\mbox{kg}\cdot\mbox{m}^{-3}$. Dado que el volumen parcial molar del etanol en la soluci\'on es $52.2\;\mbox{cm}^{3}\cdot\mbox{mol}^{-1}$, calcular el volumen parcial molar del agua. % Problema 5.2(b) de Atkins

\end{enumerate}

\end{document}
