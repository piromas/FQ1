\documentclass[a4paper,12pt]{article}
\usepackage[left=2.5cm,right=2.5cm,top=2.5cm,bottom=2.5cm]{geometry} % Adjust page margins
\usepackage{xcolor,graphicx,framed}
\usepackage[normalem]{ulem}
\usepackage{amsmath}
\usepackage{gensymb}
%\usepackage{lastpage} % Required to print the total number of pages

\begin{document}

\newcommand{\HRule}{\rule{\linewidth}{0.4mm}} % Defines a new command for the horizontal lines, change thickness here

%----------------------------------------------------------------------------------------
%	HEADING SECTIONS
%----------------------------------------------------------------------------------------

\begin{minipage}{0.7\textwidth}
\begin{flushleft} 
\textsc{Universidad del Valle de Guatemala \\
Campus Central \\
Facultad de Ciencias y Humanidades \\
Departamento de Qu\'imica \\
Segundo ciclo, 2014 \\
Fisicoqu\'imica 1 \\
}
\end{flushleft}
\end{minipage}
~
\begin{minipage}{0.2\textwidth}
\begin{flushright}
\includegraphics[scale=0.3]{Logo_UVG} % Include a department/university logo
\end{flushright}
\end{minipage}\\

%----------------------------------------------------------------------------------------
%	TITLE SECTION
%----------------------------------------------------------------------------------------

\begin{center}
\HRule \\[0.4cm]
{ \bfseries Evaluaci\'on 1}\\ % Title of your document
\HRule \\[0.4cm]
\end{center}

%----------------------------------------------------------------------------------------

\begin{enumerate}

 \item (20 pts) Indicar para cada uno de los siguientes enunciados si es verdadero o falso:
 \begin{enumerate}
  \item La primera ley de la termodin\'amica indica que $\Delta U=0$ para todo sistema.
  \item La ley de estados correspondientes indican que todos los gases tienen el mismo punto cr\'itico.
  \item La densidad es una propiedad intensiva.
  \item Una isocora es un proceso a presi\'on constante.
  \item El trabajo revesible es el trabajo m\'aximo de expansi\'on.
  \item $Z<1$ indica desviaci\'on debido al volumen de las mol\'eculas.
  \item Una pared adiab\'atica permite la transferencia de calor entre el sistema y los alrededores.
  \item Para el trabajo de expansi\'on libre se tiene $w>0$.
  \item Un sistema cerrado no permite transferencia de materia y energ\'ia entre el sistema y los alrededores.
  \item El calor y el trabajo no son funciones de estado, por lo que dependen del proceso.
 \end{enumerate}

 \item (25 pts) Una mezcla de ox\'igeno y hidr\'ogeno es analizada al hacerla pasar por \'oxido de cobre caliente y por un tubo desecante. El hidr\'ogeno reduce al $\mbox{CuO}$ seg\'un la ecuaci\'on:
$$\mbox{CuO}+\mbox{H}_2\;\Rightarrow\;\mbox{Cu}+\mbox{H}_2\mbox{O}$$
El ox\'igeno luego re-oxida al cobre que se form\'o:
$$\mbox{Cu}+\frac{1}{2}\mbox{O}_2\;\Rightarrow\;\mbox{CuO}$$
De $100.0\;\mbox{cm}^3$ de la mezcla a $25\celsius$ y $750\;\mbox{Torr}$ se obtienen $84.5\;\mbox{cm}^3$ de ox\'igeno seco a $25\celsius$ y $750\;\mbox{Torr}$ luego de pasar por el $\mbox{CuO}$ y el tubo desecante. ?`Cu\'al es la composici\'on original de la mezcla?

 \item (25 pts) Cierto gas obedece la ecuaci\'on de van der Waals con $a=0.76\;\mbox{m}^6\cdot\mbox{Pa}\cdot\mbox{mol}^{-2}$. A $288\;\mbox{K}$ y $4.0\;\mbox{MPa}$ su volumen es $4.00\times 10^{-4}\;\mbox{m}^3\cdot\mbox{mol}^{-1}$. Usando esta informaci\'on, calcular la constante $b$ de van der Waals. ?`Cu\'al es el factor de compresi\'on para este gas a dicha temperatura y presi\'on?

 \item (25 pts) Un mol de un gas ideal, $\bar{C}_V=\frac{3}{2}R$, inicialmente a $20\celsius$ y $1.0\;\mbox{MPa}$ pasa por una transformaci\'on de dos pasos. Para cada paso y para el cambio total, calcular $q$, $w$, $\Delta U$ y $\Delta H$.
 \begin{enumerate}
  \item Paso 1: expansi\'on isot\'ermica y reversible hasta el doble del volumen inicial.
  \item Paso 2: comenzando con el estado final del paso 1, manteniendo el volumen constante, la temperatura se lleva a $80\celsius$.
 \end{enumerate}

 \item (30 pts) Un llanta de carro contiene aire a $320\;\mbox{kPa}$ de presi\'on total a $20\celsius$. La v\'alvula de la llanta se retira para permitir que el aire se expanda adiab\'aticamente contra una presi\'on externa constante de $100\;\mbox{kPa}$ hasta que la presi\'on interna y externa son iguales. Tomar como capacidad calor\'ifica molar del aire $\bar{C}_V=\frac{5}{2}R$ y considerar al aire como gas ideal. Calcular la temperatura final del gas dentro de la llanta, $q$, $w$, $\Delta U$ y $\Delta H$ por mol de gas en la llanta.

\section*{Formulario}

$$R=8.314\;\mbox{J}\cdot\mbox{K}^{-1}\cdot\mbox{mol}^{-1}=0.0821\;\mbox{L}\cdot\mbox{atm}\cdot\mbox{K}^{-1}\cdot\mbox{mol}^{-1}=0.08314\;\mbox{L}\cdot\mbox{bar}\cdot\mbox{K}^{-1}\cdot\mbox{mol}^{-1}$$
$$1\;\mbox{atm}=1.01325\;\mbox{bar}=101.32\;\mbox{kPa}=760\;\mbox{torr}\quad\quad 1\;\mbox{mL}=1\;\mbox{cm}^3$$
$$T(K)=T(\celsius)+273.15\quad\quad P_J=x_JP\quad\quad P=P_A+P_B+\cdots$$
$$\left(P+\frac{an^2}{V^2}\right)(V-nb)=nRT \quad\quad Z=\frac{\bar{V}}{\bar{V}^\circ}=\frac{P\bar{V}}{RT}$$
$$\Delta U=q+w\quad\quad dw=-P_{ext}dV\quad\quad q=m\cdot c\cdot\Delta T=C\cdot\Delta T$$
$$H=U+PV\quad\quad \Delta H=\Delta U+\Delta n_g RT$$
$$C_V=\left(\frac{\partial U}{\partial T}\right)_V\quad\quad C_P=\left(\frac{\partial H}{\partial T}\right)_P$$
$$\mbox{Para un gas ideal: }\quad\quad PV=nRT\quad\quad C_P-C_V=nR$$

\end{enumerate}

\end{document}
