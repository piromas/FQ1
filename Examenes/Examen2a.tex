\documentclass[a4paper,12pt]{article}
\usepackage[left=2.5cm,right=2.5cm,top=2.5cm,bottom=2.5cm]{geometry} % Adjust page margins
\usepackage{xcolor,graphicx,framed}
\usepackage[normalem]{ulem}
\usepackage{amsmath}
\usepackage{gensymb}
\usepackage{chemmacros}
%\usepackage{lastpage} % Required to print the total number of pages

\begin{document}

\newcommand{\HRule}{\rule{\linewidth}{0.4mm}} % Defines a new command for the horizontal lines, change thickness here

%----------------------------------------------------------------------------------------
%	HEADING SECTIONS
%----------------------------------------------------------------------------------------

\begin{minipage}{0.7\textwidth}
\begin{flushleft} 
\textsc{Universidad del Valle de Guatemala \\
Campus Central \\
Facultad de Ciencias y Humanidades \\
Departamento de Qu\'imica \\
Segundo ciclo, 2014 \\
Fisicoqu\'imica 1 \\
Secci\'on 10 \\
}
\end{flushleft}
\end{minipage}
~
\begin{minipage}{0.2\textwidth}
\begin{flushright}
\includegraphics[scale=0.3]{Logo_UVG} % Include a department/university logo
\end{flushright}
\end{minipage}\\

%----------------------------------------------------------------------------------------
%	TITLE SECTION
%----------------------------------------------------------------------------------------

\begin{center}
\HRule \\[0.4cm]
{ \bfseries Evaluaci\'on 2}\\ % Title of your document
\HRule \\[0.4cm]
\end{center}

%----------------------------------------------------------------------------------------

\section*{Instrucciones}

Responder y resolver los siguientes problemas en las hojas adicionales haciendo uso de calculadora cient\'ifica, una tabla peri\'odica y el formulario individual que haya sido previamente aprobado para su uso. Cualquier actitud deshonesta ser\'a sancionada seg\'un el C\'odigo de comportamiento de los estudiantes de la Universidad.

\section*{Problemas}

\begin{enumerate}

 \item (10 pts) Indicar para cada uno de los siguientes enunciados si es verdadero o falso:
 \begin{enumerate}
  \item $\Delta U=0$ para gas ideal en un proceso isot\'ermico.
  \item $\Delta S=0$ para un proceso adiab\'atico e irreversible.
  \item $\Delta S_{tot}>0$ para un proceso espont\'aneo.
  \item $\Delta G<0$ para un proceso espont\'aneo con $V$ y $T$ constante.
  \item $\Delta H>0$ para una reacci\'on endot\'ermica cuando solo hay trabajo $P-V$ involucrado.
 \end{enumerate}

 \item (30 pts) Calcular el cambio de entrop\'ia del sistema, de los alrededores y el cambio total de entrop\'ia cuando una muestra de $21\;\mbox{g}$ de gas Arg\'on (asumir comportamiento de gas ideal) a $1.50\;\mbox{bar}$ y $298\;\mbox{K}$ incrementa su volumen de $1.20\;\mbox{dm}^3$ a $4.60\;\mbox{dm}^3$ en:
 \begin{enumerate}
  \item una expansi\'on reversible e isot\'ermica;
  \item en una expansi\'on reversible y adiab\'atica.
 \end{enumerate}

 \item (30 pts) Calcular $w$, $q$, $\Delta U$, $\Delta H$, $\Delta S$ y $\Delta G$ para un enfriamiento reversible de un gas ideal a volumen constante, $V_1$, de $P_1, V_1, T_1$ a $P_2, V_1, T_4$ seguido por una expansi\'on reversible a presi\'on constante, $P_2$, de $P_2, V_1, T_4$ a $P_2, V_2, T_1$. (Dejar indicado el resultado en t\'erminos de las variables involucradas.)

 \item (30 pts) La entalp\'ia molar de vaporizaci\'on del etanol en su punto normal de ebullici\'on ($78.29\celsius$) es $40.3\;\mbox{kJ}\cdot\mbox{mol}^{-1}$. Tomar las capacidades molares del etanol l\'iquido y gaseoso como $112.3\;\mbox{J}\cdot\mbox{K}^{-1}\cdot\mbox{mol}^{-1}$ y $65.6\;\mbox{J}\cdot\mbox{K}^{-1}\cdot\mbox{mol}^{-1}$, respectivamente, para calcular el valor de $\Delta_{vap}\bar{G}$ a $85\celsius$.

\end{enumerate}

\end{document}
