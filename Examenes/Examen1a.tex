\documentclass[a4paper,12pt]{article}
\usepackage[left=2.5cm,right=2.5cm,top=2.5cm,bottom=2.5cm]{geometry} % Adjust page margins
\usepackage{xcolor,graphicx,framed}
\usepackage[normalem]{ulem}
\usepackage{amsmath}
\usepackage{gensymb}
%\usepackage{lastpage} % Required to print the total number of pages

\begin{document}

\newcommand{\HRule}{\rule{\linewidth}{0.4mm}} % Defines a new command for the horizontal lines, change thickness here

%----------------------------------------------------------------------------------------
%	HEADING SECTIONS
%----------------------------------------------------------------------------------------

\begin{minipage}{0.7\textwidth}
\begin{flushleft} 
\textsc{Universidad del Valle de Guatemala \\
Campus Central \\
Facultad de Ciencias y Humanidades \\
Departamento de Qu\'imica \\
Segundo ciclo, 2014 \\
Fisicoqu\'imica 1 \\
}
\end{flushleft}
\end{minipage}
~
\begin{minipage}{0.2\textwidth}
\begin{flushright}
\includegraphics[scale=0.3]{Logo_UVG} % Include a department/university logo
\end{flushright}
\end{minipage}\\

%----------------------------------------------------------------------------------------
%	TITLE SECTION
%----------------------------------------------------------------------------------------

\begin{center}
\HRule \\[0.4cm]
{ \bfseries Evaluaci\'on 1}\\ % Title of your document
\HRule \\[0.4cm]
\end{center}

%----------------------------------------------------------------------------------------

\begin{enumerate}

 \item (20 pts) Indicar para cada uno de los siguientes enunciados si es verdadero o falso:
 \begin{enumerate}
  \item La energ\'ia interna y la entalp\'ia no son funciones de estado.
  \item El calor espec\'ifico es una propiedad extensiva.
  \item $\gamma=C_P/C_V=1$ para un gas ideal.
  \item Una isobara es un proceso a volumen constante.
  \item Un sistema con paredes permeables y adiab\'aticas es un sistema cerrado.
  \item $Z>1$ indica desviaci\'on devido a fuerzas intermoleculares.
  \item Para el trabajo de compresi\'on se tiene $w<0$.
  \item Un sistema aislado permite transferencia de energ\'ia entre el sistema y los alrededores.
  \item El trabajo reversible es el trabajo m\'aximo de compresi\'on.
  \item Para un proceso adiab\'atico $\Delta U=q$.
 \end{enumerate}

 \item (25 pts) Una muestra de gas contiene etano ($\mbox{C}_2\mbox{H}_6$) y butano ($\mbox{C}_4\mbox{H}_{10}$). Un recipiente de $200.0\;\mbox{cm}^3$ es llenado con el gas a una presi\'on de $100.0\;\mbox{kPa}$ a $20.0\celsius$. Si el peso del gas dentro del recipiente es $0.3846\;\mbox{g}$, ?`cu\'al es la fracci\'on molar del butano en la mezcla?

 \item (25 pts) A $27\celsius$ se tienen $10.0\;\mbox{mol}$ de $\mbox{C}_2\mbox{H}_6\mbox{(g)}$ confinados en $4.860\;\mbox{dm}^3$. Predecir la presi\'on interna del gas a partir de:
 \begin{enumerate}
  \item la ecuaci\'on de estado del gas ideal y
  \item la ecuaci\'on de estado de van der Waals.
 \end{enumerate}
Calcular el factor de compresi\'on para ambos casos. Para el etano, $a=5.507\;\mbox{dm}^6\cdot\mbox{atm}\cdot\mbox{mol}^{-2}$ y $b=0.0651\;\mbox{dm}^{3}\cdot\mbox{mol}^{-1}$.

\newpage

 \item (25 pts) Un mol de un gas ideal, $\bar{C}_V=\frac{5}{2}R$, es transformado por dos cambios de estado sucesivos:
 \begin{enumerate}
  \item A partir de $25.0\celsius$ y $100\;\mbox{kPa}$, el gas se expande isot\'ermicamente en contra de una presi\'on constante de $20\;\mbox{kPa}$ al doble de su volumen inicial.
  \item Luego del cambio anterior, el gas es enfriado a volumen constante de $25\celsius$ a $-25\celsius$.
 \end{enumerate}
Calcular $q$, $w$, $\Delta U$ y $\Delta H$ para los pasos $(a)$, $(b)$ y el proceso total $(a)+(b)$.

 \item (30 pts) Una botella a $21.0\celsius$ contiene un gas ideal bajo una presi\'on de $126.4\;\mbox{kPa}$. El tap\'on de la botella es removido, con lo que el gas se expande adiab\'aticamente contra la presi\'on externa de la atm\'osfera, $101.9\;\mbox{kPa}$. Obviamente, cierta cantidad de gas sale de la botella. Cuando la presi\'on es igual a $101.9\;\mbox{kPa}$ el tap\'on se le vuelve a colocar a la botella. El gas, que se enfri\'o en la expansi\'on adiab\'atica, se calienta lentamente hasta regresar a la temperatura de $21.0\celsius$. ?`Cu\'al es la presi\'on final en la botella,
 \begin{enumerate}
  \item si el gas es monoat\'omico, $\bar{C}_V/R=\frac{3}{2}$?
  \item si el gas es diat\'omico, $\bar{C}_V/R=\frac{5}{2}$?
 \end{enumerate}

\section*{Formulario}

$$R=8.314\;\mbox{J}\cdot\mbox{K}^{-1}\cdot\mbox{mol}^{-1}=0.0821\;\mbox{L}\cdot\mbox{atm}\cdot\mbox{K}^{-1}\cdot\mbox{mol}^{-1}=0.08314\;\mbox{L}\cdot\mbox{bar}\cdot\mbox{K}^{-1}\cdot\mbox{mol}^{-1}$$
$$1\;\mbox{atm}=1.01325\;\mbox{bar}=101.32\;\mbox{kPa}=760\;\mbox{torr}\quad\quad 1\;\mbox{mL}=1\;\mbox{cm}^3$$
$$T(K)=T(\celsius)+273.15\quad\quad P_J=x_JP\quad\quad P=P_A+P_B+\cdots$$
$$\left(P+\frac{an^2}{V^2}\right)(V-nb)=nRT \quad\quad Z=\frac{\bar{V}}{\bar{V}^\circ}=\frac{P\bar{V}}{RT}$$
$$\Delta U=q+w\quad\quad dw=-P_{ext}dV\quad\quad q=m\cdot c\cdot\Delta T=C\cdot\Delta T$$
$$H=U+PV\quad\quad \Delta H=\Delta U+\Delta n_g RT$$
$$C_V=\left(\frac{\partial U}{\partial T}\right)_V\quad\quad C_P=\left(\frac{\partial H}{\partial T}\right)_P\quad\quad \alpha=\frac{1}{\bar{V}}\left(\frac{\partial \bar{V}}{\partial T}\right)_P \quad\quad \kappa=-\frac{1}{\bar{V}}\left(\frac{\partial \bar{V}}{\partial P}\right)_T$$
$$\mbox{Para un gas ideal: }\quad\quad PV=nRT\quad\quad C_P-C_V=nR$$

\end{enumerate}

\end{document}
