\documentclass[a4paper,12pt]{article}
\usepackage[left=2.5cm,right=2.5cm,top=2.5cm,bottom=2.5cm]{geometry} % Adjust page margins
\usepackage{xcolor,graphicx,framed}
\usepackage[normalem]{ulem}
\usepackage{amsmath}
\usepackage{gensymb}
%\usepackage{lastpage} % Required to print the total number of pages

\begin{document}

\newcommand{\HRule}{\rule{\linewidth}{0.4mm}} % Defines a new command for the horizontal lines, change thickness here

%----------------------------------------------------------------------------------------
%	HEADING SECTIONS
%----------------------------------------------------------------------------------------

\begin{minipage}{0.7\textwidth}
\begin{flushleft} 
\textsc{Universidad del Valle de Guatemala \\
Campus Central \\
Facultad de Ciencias y Humanidades \\
Departamento de Qu\'imica \\
Segundo ciclo, 2014 \\
Fisicoqu\'imica 1 \\
Secci\'on: 10 \\
Profesor: Jonathan van der Henst Solis \\
Correo: jjvan@uvg.edu.gt
}
\end{flushleft}
\end{minipage}
~
\begin{minipage}{0.2\textwidth}
\begin{flushright}
\includegraphics[scale=0.4]{Logo_UVG} % Include a department/university logo
\end{flushright}
\end{minipage}\\

%----------------------------------------------------------------------------------------
%	TITLE SECTION
%----------------------------------------------------------------------------------------

\begin{center}
\HRule \\[0.4cm]
{ \bfseries Examen de diagn\'ostico}\\ % Title of your document
\HRule \\[0.4cm]
\end{center}

%----------------------------------------------------------------------------------------

\begin{enumerate}

 \item Graficar las siguientes funciones, escogiendo un sistema coordenado adecuado:
 \begin{enumerate}
  \item $y=5x-5$
  \item $r=\cos\theta$
  \item $T=\frac{1}{2}mv^2\;(\mbox{considerando a }$m$\mbox{ como una constante positiva})$
 \end{enumerate}

 \item Derivar la funci\'on $y=x^3e^{2x}$.
 \item Usando la ecuaci\'on $PV=nRT$, derivar parcialmente $P$ con respecto a $V$. 
 \item Determinar la pendiente de $y=x^2-x$ en $x=3$.
 \item Determinar si $y=4x^2-5x+4$ tiene un valor m\'aximo o m\'inimo y encontrarlo, en caso afirmativo.
 \item Evaluar la integral indefinida $\int \frac{RT}{p} dp$.
 \item Evaluar la integral definida $\int_{T_1}^{T_2}\left(a+bT+cT^2+\frac{d}{T}\right)dT$ con $a$, $b$, $c$ y $d$ constantes. 
 \item Evaluar $\int_{0}^{2\pi}d\phi$.
 \item Determinar el error de la masa molar determinada de la siguiente manera:
$$M=\frac{mRT}{PV}=\frac{(1.0339\pm 0.0007\;\text{g})(0.082057\;\text{L}\cdot\text{atm}\cdot\text{mol}^{-1}\cdot\text{K}^{-1})(274.0\pm 0.5\;\text{K})}{(1.036\pm 0.001\;\text{atm})(0.1993\pm 0.0001\;\text{L})}$$
 \item Un recipiente de $1.00$ L se llena con una mezcla de vol\'umenes iguales de ox\'igeno y di\'oxido de nitr\'ogeno a $27\celsius$ y $673$ mm Hg de presi\'on parcial. Se calienta a $420\celsius$ y una vez alcanzado el equilibrio, se encuentran $0.0404$ moles de ox\'igeno. Calcular la constante de equilibrio para el proceso
$$2NO\mbox{(g)}+O_2\mbox{(g)}\rightarrow 2NO_2{(g)}$$
y la presi\'on total de la mezcla. Asumir comportamiento de gas ideal y recordar que $760\;\mbox{mm Hg}=1\;\mbox{atm}$ y que $R=0.082057\;\text{L}\cdot\text{atm}\cdot\text{mol}^{-1}\cdot\text{K}^{-1}$. 

\end{enumerate}
 
\end{document}
