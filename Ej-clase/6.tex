\documentclass[a4paper,12pt]{article}
\usepackage[left=2.5cm,right=2.5cm,top=2.5cm,bottom=2.5cm]{geometry} % Adjust page margins
\usepackage{xcolor,graphicx,framed}
\usepackage[normalem]{ulem}
\usepackage{amsmath}
\usepackage{gensymb}
\usepackage{chemmacros}
%\usepackage{lastpage} % Required to print the total number of pages

\begin{document}

\newcommand{\HRule}{\rule{\linewidth}{0.4mm}} % Defines a new command for the horizontal lines, change thickness here

%----------------------------------------------------------------------------------------
%	HEADING SECTIONS
%----------------------------------------------------------------------------------------

\begin{minipage}{0.7\textwidth}
\begin{flushleft} 
\textsc{Universidad del Valle de Guatemala \\
Campus Central \\
Facultad de Ciencias y Humanidades \\
Departamento de Qu\'imica \\
Segundo ciclo, 2014 \\
Fisicoqu\'imica 1 \\
}
\end{flushleft}
\end{minipage}
~
\begin{minipage}{0.2\textwidth}
\begin{flushright}
\includegraphics[scale=0.3]{Logo_UVG} % Include a department/university logo
\end{flushright}
\end{minipage}\\

%----------------------------------------------------------------------------------------
%	TITLE SECTION
%----------------------------------------------------------------------------------------

\begin{center}
\HRule \\[0.4cm]
{ \bfseries Ejercicios en clase, 6}\\ % Title of your document
\HRule \\[0.4cm]
\end{center}

%----------------------------------------------------------------------------------------

\begin{enumerate}

 \item \textit{(McQuarrie 20-2)} Sea $z=(x,y)$ y $dz=xy\,dx+y^2\,dy$. Mostrar que $dz$ no es un diferencial exacto. % Problema 6-2

 \item \textit{(McQuarrie 20-6)} Calcular $w$, $q$, $\Delta U$, $\Delta H$ y $\Delta S=\int\frac{dq}{T}$ para un enfriamiento reversible de un gas ideal a volumen constante, $V_1$, de $P_1, V_1, T_1$ a $P_2, V_1, T_4$ seguido por una expansi\'on reversible a presi\'on constante, $P_2$, de $P_2, V_1, T_4$ a $P_2, V_2, T_1$. Comparar el resultado con los caminos $A$, $B+C$ y $D+E$ vistos en clase. % Problema 6-6

 \item \textit{(McQuarrie 20-11)} Para un gas ideal, en que $U$ es funci\'on s\'olo de $T$ y que obedece la ecuaci\'on de estado
$$P=\frac{RT}{\bar{V}-b}$$
donde $b$ es una constante que refleja el tama\~no de las mol\'eculas, calcular $q_{rev}$ y $\Delta S$ para los caminos $A$ y $B+C$ vistos en clase, dado un mol del gas. % Problema 6-11

 \item \textit{(McQuarrie 20-8)} Calcular el valor de $\Delta S$ cuando un mol de un gas ideal se somete a una expansi\'on reversible e isot\'ermica de $10.0\;\mbox{dm}^3$ a $20.0\;\mbox{dm}^3$. % Problema 6-8

\end{enumerate}
 
\end{document}
