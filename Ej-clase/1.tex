\documentclass[a4paper,12pt]{article}
\usepackage[left=2.5cm,right=2.5cm,top=2.5cm,bottom=2.5cm]{geometry} % Adjust page margins
\usepackage{xcolor,graphicx,framed}
\usepackage[normalem]{ulem}
\usepackage{amsmath}
\usepackage{gensymb}
%\usepackage{lastpage} % Required to print the total number of pages

\begin{document}

\newcommand{\HRule}{\rule{\linewidth}{0.4mm}} % Defines a new command for the horizontal lines, change thickness here

%----------------------------------------------------------------------------------------
%	HEADING SECTIONS
%----------------------------------------------------------------------------------------

\begin{minipage}{0.7\textwidth}
\begin{flushleft} 
\textsc{Universidad del Valle de Guatemala \\
Campus Central \\
Facultad de Ciencias y Humanidades \\
Departamento de Qu\'imica \\
Segundo ciclo, 2014 \\
Fisicoqu\'imica 1 \\
}
\end{flushleft}
\end{minipage}
~
\begin{minipage}{0.2\textwidth}
\begin{flushright}
\includegraphics[scale=0.3]{Logo_UVG} % Include a department/university logo
\end{flushright}
\end{minipage}\\

%----------------------------------------------------------------------------------------
%	TITLE SECTION
%----------------------------------------------------------------------------------------

\begin{center}
\HRule \\[0.4cm]
{ \bfseries Ejercicios en clase, 1}\\ % Title of your document
\HRule \\[0.4cm]
\end{center}

%----------------------------------------------------------------------------------------

\begin{enumerate}

 \item \textit{(McQuarrie 16-4)} ?`A qu\'e temperatura las escalas de Celsius y Fahrenheit tienen el mismo valor num\'erico? % Problema 2-4

 \item \textit{(McQuarrie 16-6)} Las investigaciones en ciencia de superficies se llevan a cabo en c\'amaras de vacio ultra alto (ultra-high vacuum chambers) que pueden mantener presiones tan bajas como $10^{-12}$ torr. ?`Cu\'antas moleculas hay en un volumen de $1.00\;\mbox{cm}^{3}$ dentro de tal aparato a $298\;\mbox{K}$? ?`Cu\'al es el correspondiente volumen molar, $\bar{V}$, a esa presi\'on y temperatura? % Problema 2-6

 \item \textit{(McQuarrie 16-11)} Se necesitan $0.3625\;\mbox{g}$ de Nitr\'ogeno para llenar un recipiente de vidrio a $298.2\;\mbox{K}$ y $0.0100\;\mbox{bar}$ de presi\'on. Si se necesitan $0.9175\;\mbox{g}$ de un gas diat\'omico homonuclear desconocido para llenar ese mismo recipiente bajo las mismas condiciones, ?`cu\'al es ese gas desconocido? % Problema 2-11

 \item \textit{(Atkins 1.6)} Un recipiente con volumen de $22.4\;\mbox{dm}^3$ contiene inicialmente $2.0\;\mbox{mol}$ de $H_2$ y $1.0\;\mbox{mol}$ de $N_2$, sumergido en un ba\~no t\'ermico a $273.15\;\mbox{K}$. Todo el $H_2$ reacciona con suficiente $N_2$ para formar $NH_3$. Calcular las presiones parciales y la presi\'on total de la mezcla final. % Problema 1.6 de Atkins

 \item \textit{(McQuarrie 16-7)} Usar la siguiente informaci\'on de un gas desconocido a $300\;\mbox{K}$ para determinar la masa molecular del gas. % Problema 2-7

\begin{center}
\begin{tabular}{c|c c c c c}
$P/bar$ & 0.1000 & 0.5000 & 1.000 & 1.01325 & 2.000 \\\hline
$\rho/g\cdot L^{-1}$ & 0.1771 & 0.8909 & 1.796 & 1.820 & 3.652 
\end{tabular}
\end{center}

\end{enumerate}
 
\end{document}
