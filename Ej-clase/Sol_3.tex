\documentclass[a4paper,12pt]{article}
\usepackage[left=2.5cm,right=2.5cm,top=2.5cm,bottom=2.5cm]{geometry} % Adjust page margins
\usepackage{xcolor,graphicx,framed}
\usepackage[normalem]{ulem}
\usepackage{amsmath}
\usepackage{gensymb}
%\usepackage{lastpage} % Required to print the total number of pages

\begin{document}

\newcommand{\HRule}{\rule{\linewidth}{0.4mm}} % Defines a new command for the horizontal lines, change thickness here

%----------------------------------------------------------------------------------------
%	HEADING SECTIONS
%----------------------------------------------------------------------------------------

\begin{minipage}{0.7\textwidth}
\begin{flushleft} 
\textsc{Universidad del Valle de Guatemala \\
Campus Central \\
Facultad de Ciencias y Humanidades \\
Departamento de Qu\'imica \\
Segundo ciclo, 2014 \\
Fisicoqu\'imica 1 \\
}
\end{flushleft}
\end{minipage}
~
\begin{minipage}{0.2\textwidth}
\begin{flushright}
\includegraphics[scale=0.3]{Logo_UVG} % Include a department/university logo
\end{flushright}
\end{minipage}\\

%----------------------------------------------------------------------------------------
%	TITLE SECTION
%----------------------------------------------------------------------------------------

\begin{center}
\HRule \\[0.4cm]
{ \bfseries Soluciones propuestas a los ejercicios en clase, 3}\\ % Title of your document
\HRule \\[0.4cm]
\end{center}

%----------------------------------------------------------------------------------------

\begin{enumerate}

 \item \textbf{\textit{(Atkins 2.5(a))} Una muestra de $4.50\;\mbox{g}$ de metano ocupa $12.7\;\mbox{dm}^3$ a $310\;\mbox{K}$.} 
 \begin{enumerate} 
   \item \textbf{Calcular el trabajo realizado cuando el gas se expande isot\'ermicamente en contra de una presi\'on externa constante de $200\;\mbox{torr}$ hasta que su volumen se incrementa por $3.3\;\mbox{dm}^3$.}

Como es expansi\'on isot\'ermica a presi\'on externa constante, se puede calcular el trabajo de la siguiente manera:
\begin{align*}
w& =-P_{ext}\Delta V=-\left(200\;\mbox{torr}\left(\frac{133.322\;\mbox{Pa}}{1\;\mbox{torr}}\right)\right)\left(3.3\;\mbox{dm}^3\left(\frac{1\;\mbox{m}^3}{1000\;\mbox{dm}^3}\right)\right)\\
&=-88.0\;\mbox{J}
\end{align*}

   \item \textbf{Calcular el trabajo que se realizar\'ia si la misma expansi\'on se realiza reversiblemente.}

De primero, determinamos el n\'umero de moles:
$$n=\frac{m}{M}=\frac{4.50\;\mbox{g}}{16.04\;\mbox{g}\cdot\mbox{mol}^{-1}}=0.2805\;\mbox{mol}$$
Y el volumen final: $V_f=12.7\;\mbox{dm}^3+3.3\;\mbox{dm}^3=16.0\;\mbox{dm}^3$.
Asumiendo el comportamiento de un gas ideal, el trabajo isot\'ermico reversible puede ser calculado de la siguiente manera:
\begin{align*}
w_{rev}&=-nRT\ln\frac{V_f}{V_i}=-(0.2805\;\mbox{mol})(8.314\;\mbox{J}\cdot\mbox{K}^{-1}\cdot\mbox{mol}^{-1})(310\;\mbox{K})\ln\frac{16.0\;\mbox{dm}^3}{12.7\;\mbox{dm}^3} \\
&=-167\;\mbox{J}
\end{align*}
Observar que el trabajo de expansi\'on reversible es m\'as grande en magnitud que el no reversible (el trabajo reversible de expansi\'on es el valor m\'aximo de expansi\'on).

 \end{enumerate}

 \item \textbf{\textit{(McQuarrie 19-5)} Calcular el trabajo involucrado cuando un mol de un gas ideal se expande reversiblemente de $20.0\;\mbox{dm}^3$ a $40.0\;\mbox{dm}^3$, a una temperatura constante de $300\;\mbox{K}$.} % Problema 5-5

El trabajo de expansi\'on isot\'ermico revesible de un gas ideal viene dado por:
$$w_{rev}=-nRT\ln\frac{V_f}{V_i}=-(1\;\mbox{mol})(8.314\;\mbox{J}\cdot\mbox{K}^{-1}\cdot\mbox{mol}^{-1})(300\;\mbox{K})\ln\frac{40.0\;\mbox{dm}^3}{20.0\;\mbox{dm}^3}=-1.73\;\mbox{kJ}$$

 \item \textbf{\textit{(McQuarrie 19-4)} Calcular el trabajo involucrado cuando un mol de un gas ideal se comprime reversiblemente de $1.00\;\mbox{bar}$ a $5.00\;\mbox{bar}$, a una temperatura constante de $300\;\mbox{K}$.} % Problema 5-4

Usando la expresi\'on para el trabajo de expansi\'on isot\'ermico reversible de un gas ideal, en t\'erminos de las presiones:
\begin{align*}
w_{rev}&=-nRT\ln\frac{V_f}{V_i}=nRT\ln\frac{P_f}{P_i}=(1\;\mbox{mol})(8.314\;\mbox{J}\cdot\mbox{K}^{-1}\cdot\mbox{mol}^{-1})(300\;\mbox{K})\ln\frac{5.00\;\mbox{bar}}{1.00\;\mbox{bar}}\\
&=4.014\;\mbox{kJ}
\end{align*}

 \item \textbf{\textit{(McQuarrie 19-10)} Usar la ecuaci\'on de van der Waals para calcular el trabajo involucrado en la expansi\'on reversible isot\'ermica a $300\;\mbox{K}$ de un mol de $CH_4(g)$ desde $1.00\;\mbox{dm}^3\cdot\mbox{mol}^{-1}$ a $5.00\;\mbox{dm}^3\cdot\mbox{mol}^{-1}$ (para el metano tenemos $a=2.3026\;\mbox{dm}^6\cdot\mbox{bar}\cdot\mbox{mol}^{-2}$ y $b=0.043067\;\mbox{dm}^3\cdot\mbox{mol}^{-1}$).} % Problema 5-10

Como tenemos 1 mol de gas y sabiendo que $V=\bar{V}\cdot n$, entonces $V_i=(1.00\;\mbox{dm}^3\cdot\mbox{mol}^{-1})(1\;\mbox{mol})=1.00\;\mbox{dm}^3$ y $V_f=(5.00\;\mbox{dm}^3\cdot\mbox{mol}^{-1})(1\;\mbox{mol})=5.00\;\mbox{dm}^3$. El trabajo de expansi\'on isot\'ermico reversible para un gas seg\'un la ecuaci\'on de van der Waals viene dado por: 
\begin{align*}
w&=-nRT\ln\frac{(V_f-nb)}{(V_i-nb)}-an^2\left(\frac{1}{V_f}-\frac{1}{V_i}\right) \\
&=-(1\;\mbox{mol})(8.314\;\mbox{J}\cdot\mbox{K}^{-1}\cdot\mbox{mol}^{-1})(300\;\mbox{K})\ln\frac{(5.00\;\mbox{dm}^3-(1\;\mbox{mol})(0.043067\;\mbox{dm}^3\cdot\mbox{mol}^{-1}))}{(1.00\;\mbox{dm}^3-(1\;\mbox{mol})(0.043067\;\mbox{dm}^3\cdot\mbox{mol}^{-1}))} \\
& -(2.3026\;\mbox{dm}^6\cdot\mbox{bar}\cdot\mbox{mol}^{-2})(1\;\mbox{mol})^2\left(\frac{1}{5.00\;\mbox{dm}^3}-\frac{1}{1.00\;\mbox{dm}^3}\right)\left(\frac{10^5\;\mbox{Pa}}{1\;\mbox{bar}}\right)\left(\frac{1\;\mbox{m}^3}{1000\;\mbox{dm}^3}\right) \\
&=-3.92\;\mbox{kJ}
\end{align*}

 \item \textbf{\textit{(Atkins 2.35)} Rearreglar la ecuaci\'on de van der Waals para obtener una expresi\'on de $T$ en funci\'on de $P$ y $V$ (con $n$ constante). Calcular $(\partial T/\partial P)_{V,n}$. Verificar si se cumple la siguiente igualdad:
$$\left(\frac{\partial T}{\partial P}\right)_{V,n}=\frac{1}{\left(\frac{\partial P}{\partial T}\right)_{V,n}}$$} % Problema 2.35 de Atkins

Expresando $T$ en funci\'on de $P$ y $V$ (con $n$ constante) se obtiene:
$$\left(P+\frac{an^2}{V^2}\right)\left(V-nb\right)=nRT\rightarrow T=\frac{1}{nR}\left(P+\frac{an^2}{V^2}\right)\left(V-nb\right)$$
Lo que nos permite tomar la derivada parcial f\'acilmente:
$$\left(\frac{\partial T}{\partial P}\right)_{V,n}=\left(\frac{\partial }{\partial P}\right)_{V,n}\left[\frac{1}{nR}\left(P+\frac{an^2}{V^2}\right)\left(V-nb\right)\right]=\frac{V-nb}{nR}$$
Por otra parte, para verificar la relaci\'on podemos despejar $P$ en funci\'on de $V$ y $T$ (con $n$ constante):
$$\left(P+\frac{an^2}{V^2}\right)\left(V-nb\right)=nRT\rightarrow P=\frac{nRT}{V-nb}-\frac{an^2}{V^2}$$
Para determinar la derivada parcial:
$$\left(\frac{\partial P}{\partial T}\right)_{V,n}=\left(\frac{\partial }{\partial T}\right)_{V,n}\left(\frac{nRT}{V-nb}-\frac{an^2}{V^2}\right)=\frac{nR}{V-nb}$$
Por lo que la relaci\'on dada se cumple:
$$\left(\frac{\partial T}{\partial P}\right)_{V,n}=\frac{V-nb}{nR}=\frac{1}{(nR)/(V-nb)}=\frac{1}{\left(\frac{\partial P}{\partial T}\right)_{V,n}}$$

\end{enumerate}

\end{document}
