\documentclass[a4paper,12pt]{article}
\usepackage[left=2.5cm,right=2.5cm,top=2.5cm,bottom=2.5cm]{geometry} % Adjust page margins
\usepackage{xcolor,graphicx,framed}
\usepackage[normalem]{ulem}
\usepackage{amsmath}
\usepackage{gensymb}
%\usepackage{lastpage} % Required to print the total number of pages

\begin{document}

\newcommand{\HRule}{\rule{\linewidth}{0.4mm}} % Defines a new command for the horizontal lines, change thickness here

%----------------------------------------------------------------------------------------
%	HEADING SECTIONS
%----------------------------------------------------------------------------------------

\begin{minipage}{0.7\textwidth}
\begin{flushleft} 
\textsc{Universidad del Valle de Guatemala \\
Campus Central \\
Facultad de Ciencias y Humanidades \\
Departamento de Qu\'imica \\
Segundo ciclo, 2014 \\
Fisicoqu\'imica 1 \\
}
\end{flushleft}
\end{minipage}
~
\begin{minipage}{0.2\textwidth}
\begin{flushright}
\includegraphics[scale=0.3]{Logo_UVG} % Include a department/university logo
\end{flushright}
\end{minipage}\\

%----------------------------------------------------------------------------------------
%	TITLE SECTION
%----------------------------------------------------------------------------------------

\begin{center}
\HRule \\[0.4cm]
{ \bfseries Ejercicios en clase, 3}\\ % Title of your document
\HRule \\[0.4cm]
\end{center}

%----------------------------------------------------------------------------------------

\begin{enumerate}

 \item \textit{(Atkins 2.5(a))} Una muestra de $4.50\;\mbox{g}$ de metano ocupa $12.7\;\mbox{dm}^3$ a $310\;\mbox{K}$. 
 \begin{enumerate} 
   \item Calcular el trabajo realizado cuando el gas se expande isot\'ermicamente en contra de una presi\'on externa constante de $200\;\mbox{torr}$ hasta que su volumen se incrementa por $3.3\;\mbox{dm}^3$.
   \item Calcular el trabajo que se realizar\'ia si la misma expansi\'on se realiza reversiblemente.
 \end{enumerate} % Ejercicio 2.5(a) de Atkins

 \item \textit{(McQuarrie 19-5)} Calcular el trabajo involucrado cuando un mol de un gas ideal se expande reversiblemente de $20.0\;\mbox{dm}^3$ a $40.0\;\mbox{dm}^3$, a una temperatura constante de $300\;\mbox{K}$. % Problema 5-5

 \item \textit{(McQuarrie 19-4)} Calcular el trabajo involucrado cuando un mol de un gas ideal se comprime reversiblemente de $1.00\;\mbox{bar}$ a $5.00\;\mbox{bar}$, a una temperatura constante de $300\;\mbox{K}$. % Problema 5-4

 \item \textit{(McQuarrie 19-10)} Usar la ecuaci\'on de van der Waals para calcular el trabajo involucrado en la expansi\'on reversible isot\'ermica a $300\;\mbox{K}$ de un mol de $CH_4(g)$ desde $1.00\;\mbox{dm}^3\cdot\mbox{mol}^{-1}$ a $5.00\;\mbox{dm}^3\cdot\mbox{mol}^{-1}$ (para el metano tenemos $a=2.3026\;\mbox{dm}^6\cdot\mbox{bar}\cdot\mbox{mol}^{-2}$ y $b=0.043067\;\mbox{dm}^3\cdot\mbox{mol}^{-1}$). % Problema 5-10

 \item \textit{(Atkins 2.35)} Rearreglar la ecuaci\'on de van der Waals para obtener una expresi\'on de $T$ en funci\'on de $P$ y $V$ (con $n$ constante). Calcular $(\partial T/\partial P)_{V,n}$. Verificar si se cumple la siguiente igualdad:
$$\left(\frac{\partial T}{\partial P}\right)_{V,n}=\frac{1}{\left(\frac{\partial P}{\partial T}\right)_{V,n}}$$ % Problema 2.35 de Atkins

\end{enumerate}

\end{document}
