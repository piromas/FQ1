\documentclass[a4paper,12pt]{article}
\usepackage[left=2.5cm,right=2.5cm,top=2.5cm,bottom=2.5cm]{geometry} % Adjust page margins
\usepackage{xcolor,graphicx,framed}
\usepackage[normalem]{ulem}
\usepackage{amsmath}
\usepackage{cases}
\usepackage{gensymb}
\usepackage{chemmacros}
\setlength{\extrarowheight}{0.4cm}

\begin{document}

\newcommand{\HRule}{\rule{\linewidth}{0.4mm}} % Defines a new command for the horizontal lines, change thickness here

%----------------------------------------------------------------------------------------
%	HEADING SECTIONS
%----------------------------------------------------------------------------------------

\begin{minipage}{0.7\textwidth}
\begin{flushleft} 
\textsc{Universidad del Valle de Guatemala \\
Campus Central \\
Facultad de Ciencias y Humanidades \\
Departamento de Qu\'imica \\
Segundo ciclo, 2014 \\
Fisicoqu\'imica 1 \\
}
\end{flushleft}
\end{minipage}
~
\begin{minipage}{0.2\textwidth}
\begin{flushright}
\includegraphics[scale=0.3]{Logo_UVG} % Include a department/university logo
\end{flushright}
\end{minipage}\\

%----------------------------------------------------------------------------------------
%	TITLE SECTION
%----------------------------------------------------------------------------------------

\begin{center}
\HRule \\[0.4cm]
{ \bfseries Ejercicios en clase, 12}\\ % Title of your document
\HRule \\[0.4cm]
\end{center}

%----------------------------------------------------------------------------------------

\begin{enumerate}

 \item \textit{(Chang 7.55)} Los vol\'umenes molares parciales de una soluci\'on benceno-tetracloruro de carbono a $25\celsius$ a una fracci\'on molar de 0.5 son: $V_b=0.106\;\mbox{L}\cdot\mbox{mol}^{-1}$ y $V_c=0.100\;\mbox{L}\cdot\mbox{mol}^{-1}$, respectivamente, donde los sub\'indice b y c denotan $\mbox{C}_6\mbox{H}_6$ y $\mbox{CCl}_4$.
 \begin{enumerate}
  \item ?`Cu\'al es el volumen de la soluci\'on formada por un mol de cada uno de ellos?
  \item Dado que los vol\'umenes molares son $\mbox{C}_6\mbox{H}_6=0.089\;\mbox{L}\cdot\mbox{mol}^{-1}$ y $\mbox{CCl}_4=0.097\;\mbox{L}\cdot\mbox{mol}^{-1}$, ?`cu\'al es el cambio de volumen al mezclar $1\;\mbox{mol}$ de cada uno de ellos ($\mbox{C}_6\mbox{H}_6$ y $\mbox{CCl}_4$)?
  \item ?`Qu\'e puede deducir acerca de la naturaleza de las fuerzas intermoleculares entre el $\mbox{C}_6\mbox{H}_6$ y el $\mbox{CCl}_4$?
 \end{enumerate} % Problema 7.55 de Chang

 \item \textit{(Chang 7.58)} Suponga que se mezclan $2.6\;\mbox{moles}$ de $\mbox{He}$ a $0.80\;\mbox{atm}$ y $25\celsius$ con $4.1\;\mbox{moles}$ de $\mbox{Ne}$ a $2.7\;\mbox{atm}$ y $25\celsius$. Calcule el cambio de la energ\'ia de Gibbs del proceso. Suponga un comportamiento ideal. % Problema 7.58 de Chang

 \item \textit{(Chang 7.11)} A $25\celsius$ y $1\;\mbox{atm}$ de presi\'on, las entrop\'ias absolutas de la tercera ley para el metano y el etano son $186.19\;\mbox{J}\cdot\mbox{K}^{-1}\cdot\mbox{mol}^{-1}$ y $229.49\;\mbox{J}\cdot\mbox{K}^{-1}\cdot\mbox{mol}^{-1}$, respectivamente, en la fase gaseosa. Calcule la entrop\'ia absoluta de tercera ley para una soluci\'on que contiene $1\;\mbox{mol}$ de cada gas. Suponga un comportamiento ideal. % Problema 7.11 de Chang

 \item \textit{(Chang 7.14)} La constante de la ley de Henry para ox\'igeno en el agua a $25\celsius$ es de $773\;\mbox{atm}\cdot\mbox{mol}^{-1}\cdot\mbox{kg}$ de agua. Calcule la molalidad del ox\'igeno en el agua a una presi\'on parcial de $0.20\;\mbox{atm}$. Suponiendo que la solubilidad del ox\'igeno en la sangre a $37\celsius$ es aproximadamente la misma que en agua a $25\celsius$, comente sobre la esperanza de sobrevivencia humana sin mol\'eculas de hemoglobina. (El volumen total de sangre en el cuerpo humano es aproximadamente de $5\;\mbox{L}$.) % Problema 7.13 de Chang

 \item \textit{(McQuarrie 24-19)} El tetraclorometano y el tricloroetileno forman una solución esencialmente ideal a $40\celsius$ para cualquier concentración. Las presiones de vapor a $40\celsius$ son $214\;\mbox{torr}$ y $138\;\mbox{torr}$, respectivamente. La presi\'on total en t\'erminos de $x_2$ viene dada por:
$$P=x_1P_1^\star+x_2P_2^\star=(1-x_2)P_1^\star+x_2P_2^\star=P_1^\star+x_2(P_2^\star-P_1^\star)$$
Determinar $y_2$ en funci\'on de $x_2$ (incluyendo las constantes $P_1^\star$ y $P_2^\star$). Despejar $x_2$ y sustituirlo en la ecuaci\'on anterior. Usar este resultado para construir el diagrama de presi\'on-composici\'on de esta soluci\'on. % Problema 10-19

\end{enumerate}
 
\end{document}
