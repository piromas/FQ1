\documentclass[a4paper,12pt]{article}
\usepackage[left=2.5cm,right=2.5cm,top=2.5cm,bottom=2.5cm]{geometry} % Adjust page margins
\usepackage{xcolor,graphicx,framed}
\usepackage[normalem]{ulem}
\usepackage{amsmath}
\usepackage{cases}
\usepackage{gensymb}
\usepackage{chemmacros}
\setlength{\extrarowheight}{0.4cm}

\begin{document}

\newcommand{\HRule}{\rule{\linewidth}{0.4mm}} % Defines a new command for the horizontal lines, change thickness here

%----------------------------------------------------------------------------------------
%	HEADING SECTIONS
%----------------------------------------------------------------------------------------

\begin{minipage}{0.7\textwidth}
\begin{flushleft} 
\textsc{Universidad del Valle de Guatemala \\
Campus Central \\
Facultad de Ciencias y Humanidades \\
Departamento de Qu\'imica \\
Segundo ciclo, 2014 \\
Fisicoqu\'imica 1 \\
}
\end{flushleft}
\end{minipage}
~
\begin{minipage}{0.2\textwidth}
\begin{flushright}
\includegraphics[scale=0.3]{Logo_UVG} % Include a department/university logo
\end{flushright}
\end{minipage}\\

%----------------------------------------------------------------------------------------
%	TITLE SECTION
%----------------------------------------------------------------------------------------

\begin{center}
\HRule \\[0.4cm]
{ \bfseries Soluciones propuestas a los ejercicios en clase, 14}\\ % Title of your document
\HRule \\[0.4cm]
\end{center}

%----------------------------------------------------------------------------------------

\begin{enumerate}

 \item \textbf{\textit{(McQuarrie 25-27)} Una soluci\'on que contiene $1.470\;\mbox{g}$ de diclorobenceno en $50.00\;\mbox{g}$ de benceno ebulle a $80.60\,\celsius$ a una presi\'on de $1.00\;\mbox{bar}$. El punto de ebullici\'on est\'andar del benceno puro es $80.09\,\celsius$ y la entalp\'ia molar de vaporizaci\'on del benceno puro es $32.0\;\mbox{kJ}\cdot\mbox{mol}^{-1}$. Determinar la masa molecular del diclorobenceno a partir de esta informaci\'on.} % Problema 11-27



 \item \textbf{\textit{(Chang 7.39)} La lisozima extra\'ida del huevo blanco de gallina tiene una masa molar de $13\,930\;\mbox{g}\cdot\mbox{mol}^{-1}$. Se disuelven exactamente $0.1\;\mbox{g}$ de esta prote\'ina en $50\;\mbox{g}$ de agua a $298\;\mbox{K}$. Calcular la reducci\'on de la presi\'on de vapor ($\Delta P$), el descenso del punto de congelaci\'on ($\Delta T$), la elevaci\'on del punto de ebullici\'on $\Delta T$ y la presi\'on osm\'otica ($\Pi$) de esta soluci\'on. Para el agua pura, la presi\'on de vapor a $298\;\mbox{K}$ es de $23.76\;\mbox{mmHg}$, $K_f=1.86\;\mbox{K}\cdot\mbox{kg}\cdot\mbox{mol}^{-1}$ y $K_b=0.51\;\mbox{K}\cdot\mbox{kg}\cdot\mbox{mol}^{-1}$.} % Problema 7.39 de Chang



 \item \textbf{\textit{(McQuarrie 26-2)} Escribir la expresi\'on de la constante de equilibrio para la reacci\'on que est\'a descrita por la ecuaci\'on qu\'imica:}
$$2\,\mbox{SO}_2\mbox{(g)}+\mbox{O}_2\mbox{(g)}\;\ch{ <=> }\;2\,\mbox{SO}_3\mbox{(g)}$$
\textbf{Comparar el resultado con lo que se obtiene si la reacci\'on estuviera representada por:}
$$\mbox{SO}_2\mbox{(g)}+\frac{1}{2}\,\mbox{O}_2\mbox{(g)}\;\ch{ <=> }\;\mbox{SO}_3\mbox{(g)}$$ % Problema 12-2



 \item \textbf{\textit{(McQuarrie 26-10)} Usando una tabla de datos termodin\'amicos, calcular $\Delta_{rxn}G^\standardstate(T)$ y $K_P(T)$ a $25\,\celsius$ para:}
 \begin{enumerate}
  \item $\mbox{N}_2\mbox{O}_4\mbox{(g)}\;\ch{ <=> }\;2\,\mbox{NO}_2\mbox{(g)}$


  \item $\mbox{H}_2\mbox{(g)}+\mbox{I}_2\mbox{(g)}\;\ch{ <=> }\;2\,\mbox{HI(g)}$


  \item $3\,\mbox{H}_2\mbox{(g)}+\mbox{N}_2\mbox{(g)}\;\ch{ <=> }\;2\,\mbox{NH}_3\mbox{(g)}$


 \end{enumerate} % Problema 12-10

 \item \textbf{\textit{(McQuarrie 26-21)} Suponer que se tiene una mezcla de los gases $\mbox{H}_2\mbox{(g)}$, $\mbox{CO}_2\mbox{(g)}$, $\mbox{CO(g)}$ y $\mbox{H}_2\mbox{O(g)}$ a $1260\;\mbox{K}$, con $P_{H_2}=0.55\;\mbox{bar}$, $P_{CO_2}=0.20\;\mbox{bar}$, $P_{CO}=1.25\;\mbox{bar}$ y $P_{H_2O}=0.10\;\mbox{bar}$. ?`Est\'a en equilibrio la reacci\'on representada por la siguiente ecuaci\'on}
$$\mbox{H}_2\mbox{(g)}+\mbox{CO}_2\mbox{(g)}\;\ch{ <=> }\;\mbox{CO(g)}+\mbox{H}_2\mbox{O(g)}\;\;\;\;K_P=1.59$$
\textbf{bajo las condiciones anteriores? Si no est\'a en equilibrio, ?`en qu\'e direcci\'on va a proceder para alcanzar el equilibrio?} % Problema 12-21



 \item \textbf{\textit{(Chang 9.8)} Considere la descomposici\'on t\'ermica del $\mbox{CaCO}_3$:}
$$\mbox{CaCO}_3\mbox{(s)}\;\ch{ <=> }\;\mbox{CaO(s)}+\mbox{CO}_2\mbox{(g)}$$
\textbf{Las presiones de vapor en equilibrio del $\mbox{CO}_2$ son $22.6\;\mbox{mmHg}$ a $700\,\celsius$ y de $1\,829\;\mbox{mmHg}$ a $950\,\celsius$. Calcular la entalp\'ia est\'andar de la reacci\'on.} % Problema 9.8 de Chang



\end{enumerate}
 
\end{document}
