\documentclass[a4paper,12pt]{article}
\usepackage[left=2.5cm,right=2.5cm,top=2.5cm,bottom=2.5cm]{geometry} % Adjust page margins
\usepackage{xcolor,graphicx,framed}
\usepackage[normalem]{ulem}
\usepackage{amsmath}
\usepackage{gensymb}
\usepackage{chemmacros}
%\usepackage{lastpage} % Required to print the total number of pages

\begin{document}

\newcommand{\HRule}{\rule{\linewidth}{0.4mm}} % Defines a new command for the horizontal lines, change thickness here

%----------------------------------------------------------------------------------------
%	HEADING SECTIONS
%----------------------------------------------------------------------------------------

\begin{minipage}{0.7\textwidth}
\begin{flushleft} 
\textsc{Universidad del Valle de Guatemala \\
Campus Central \\
Facultad de Ciencias y Humanidades \\
Departamento de Qu\'imica \\
Segundo ciclo, 2014 \\
Fisicoqu\'imica 1 \\
}
\end{flushleft}
\end{minipage}
~
\begin{minipage}{0.2\textwidth}
\begin{flushright}
\includegraphics[scale=0.3]{Logo_UVG} % Include a department/university logo
\end{flushright}
\end{minipage}\\

%----------------------------------------------------------------------------------------
%	TITLE SECTION
%----------------------------------------------------------------------------------------

\begin{center}
\HRule \\[0.4cm]
{ \bfseries Ejercicios en clase, 5}\\ % Title of your document
\HRule \\[0.4cm]
\end{center}

%----------------------------------------------------------------------------------------

\begin{enumerate}

 \item \textit{(McQuarrie 19-37)} Las entalp\'ias molares estandar de combusti\'on para los is\'omeros $m\mbox{-xileno}$ y $p\mbox{-xileno}$ son $-4553.9\;\mbox{kJ}\cdot\mbox{mol}^{-1}$ y $-4556.8\;\mbox{kJ}\cdot\mbox{mol}^{-1}$, respectivamente. Usar estos datos, junto con la ley de Hess, para calcular el valor de $\Delta_rH^\standardstate$ para la reacci\'on:
$$m\mbox{-xileno}\rightarrow p\mbox{-xileno}$$ % Problema 5-37

 \item \textit{(McQuarrie 19-42)} Usar los siguientes datos para calcular el valor de $\Delta_{vap}H^\standardstate$ del agua a $298\;\mbox{K}$: $\Delta_{vap}H^\standardstate\mbox{ a 373 K}=40.7\;\mbox{kJ}\cdot\mbox{mol}^{-1}$; $\bar{C}_P\mbox{(l)}=75.2\;\mbox{J}\cdot\mbox{mol}^{-1}\cdot\mbox{K}^{-1}$; $\bar{C}_P\mbox{(g)}=33.6\;\mbox{J}\cdot\mbox{mol}^{-1}\cdot\mbox{K}^{-1}$. Comparar el resultado con valor que se obtendr\'ia a partir de $\Delta_{f}H^\standardstate=-285.83\;\mbox{kJ}\cdot\mbox{mol}^{-1}$ para $\mbox{H}_2\mbox{O(l)}$ y $\Delta_{f}H^\standardstate=-241.8\;\mbox{kJ}\cdot\mbox{mol}^{-1}$ para $\mbox{H}_2\mbox{O(g)}$, a $25\celsius$. % Problema 5-42

 \item \textit{(McQuarrie 19-22)} Un mol de etano a $25\celsius$ y una atm\'osfera es calentado hasta $1200\celsius$ a presi\'on constante. Asumiendo un comportamiento ideal, calcular los valores de $w$, $q$, $\Delta U$ y $\Delta H$ si su capacidad calor\'ifica molar est\'a dada por:
$$\bar{C}_P/R=0.06436+(2.137\times 10^{-2}\;\mbox{K}^{-1})T-(8.263\times 10^{-6}\;\mbox{K}^{-2})T^2+(1.024\times 10^{-9}\;\mbox{K}^{-3})T^3$$
para el rango de temperatura dado. Repetir el c\'alculo para un proceso a volumen constante. %Problema 5-22

 \item \textit{(McQuarrie 19-34)} Dados los siguientes datos del sodio, graficar $\bar{H}(T)-\bar{H}(0)$ en funci\'on de $T$ para el sodio: punto de fusi\'on, $361\;\mbox{K}$; punto de ebullici\'on, $1156\;\mbox{K}$; $\Delta_{fus}H^\standardstate=2.60\;\mbox{kJ}\cdot\mbox{mol}^{-1}$; $\Delta_{vap}H^\standardstate=97.4\;\mbox{kJ}\cdot\mbox{mol}^{-1}$; $\bar{C}_P(s)=28.2\;\mbox{J}\cdot\mbox{mol}^{-1}\cdot\mbox{K}^{-1}$; $\bar{C}_P\mbox{(l)}=32.7\;\mbox{J}\cdot\mbox{mol}^{-1}\cdot\mbox{K}^{-1}$; $\bar{C}_P\mbox{(g)}=20.8\;\mbox{J}\cdot\mbox{mol}^{-1}\cdot\mbox{K}^{-1}$. % Problema 5-34

 \item \textit{(McQuarrie 19-29)} A partir de $H=U+PV$, mostrar que:
$$\left(\frac{\partial U}{\partial T}\right)_P=C_P-P\left(\frac{\partial V}{\partial T}\right)_P$$
Interpretar f\'isicamente este resultado. % Problema 5-29

\end{enumerate}
 
\end{document}
