\documentclass[a4paper,12pt]{article}
\usepackage[left=2.5cm,right=2.5cm,top=2.5cm,bottom=2.5cm]{geometry} % Adjust page margins
\usepackage{xcolor,graphicx,framed}
\usepackage[normalem]{ulem}
\usepackage{amsmath}
\usepackage{gensymb}
%\usepackage{lastpage} % Required to print the total number of pages

\begin{document}

\newcommand{\HRule}{\rule{\linewidth}{0.4mm}} % Defines a new command for the horizontal lines, change thickness here

%----------------------------------------------------------------------------------------
%	HEADING SECTIONS
%----------------------------------------------------------------------------------------

\begin{minipage}{0.7\textwidth}
\begin{flushleft} 
\textsc{Universidad del Valle de Guatemala \\
Campus Central \\
Facultad de Ciencias y Humanidades \\
Departamento de Qu\'imica \\
Segundo ciclo, 2014 \\
Fisicoqu\'imica 1 \\
}
\end{flushleft}
\end{minipage}
~
\begin{minipage}{0.2\textwidth}
\begin{flushright}
\includegraphics[scale=0.3]{Logo_UVG} % Include a department/university logo
\end{flushright}
\end{minipage}\\

%----------------------------------------------------------------------------------------
%	TITLE SECTION
%----------------------------------------------------------------------------------------

\begin{center}
\HRule \\[0.4cm]
{ \bfseries Ejercicios en clase, 4}\\ % Title of your document
\HRule \\[0.4cm]
\end{center}

%----------------------------------------------------------------------------------------

\begin{enumerate}

 \item \textit{(McQuarrie 19-25)} Una muestra de $25.0\;\mbox{g}$ de cobre a $363\;\mbox{K}$ es introducida en $100.0\;\mbox{g}$ de agua a $293\;\mbox{K}$. El cobre y el agua r\'apidamente llegan a la misma temperatura por el proceso de tranferencia de calor del cobre al agua. Calcular la temperatura final del agua. La capacidad calor\'ifica molar del cobre es $24.5\;\mbox{J}\cdot\mbox{K}^{-1}\cdot\mbox{mol}^{-1}$. % Problema 5-25

 \item \textit{(McQuarrie 19-20)} Cierta cantidad de $N_2(g)$ a $298\;\mbox{K}$ se comprime reversible y adiab\'aticamente de $20.0\;\mbox{dm}^3$ a $5.00\;\mbox{dm}^3$. Asumiendo un comportamiento ideal, calcular la temperatura final del $N_2(g)$. Tomar $\bar{C}_V=5R/2$. % Problema 5-20

 \item \textit{(McQuarrie 19-23)} El valor de $\Delta H$ a $25\celsius$ y un $bar$ es $+290.9\;\mbox{kJ}$ para la reacci\'on
$$2\,ZnO(s)+2\,S(s)\rightarrow 2\,ZnS(s)+O_2(g)$$
Asumiendo un comportamiento ideal, calcular el valor de $\Delta U$ para esta reacci\'on. % Problema 5-23

 \item \textit{(McQuarrie 23-12)} Un mol de un gas ideal monoat\'omico que est\'a inicialmente a una presi\'on de $2.00\;\mbox{bar}$ y una temperatura de $273\;\mbox{K}$ es llevado a una presi\'on final de $4.00\;\mbox{bar}$ por un camino reversible definido por $P/V=\mbox{constante}$. Calcular los valores de $\Delta U$, $\Delta H$, $q$ y $w$ para este proceso. Tomar $\bar{C}_V=12.5\;\mbox{J}\cdot\mbox{mol}^{-1}\cdot\mbox{K}^{-1}$. % Problema 5-12

 \item \textit{(McQuarrie 19-33)} Derivar $H=U+PV$ con respecto a $V$ a temperatura constante para mostrar que $(\partial H/\partial V)_T=0$ para un gas ideal (asumir que para un gas ideal, $U$ solo depende de $T$). % Problema 5-33

\end{enumerate}

\end{document}
