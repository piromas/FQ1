\documentclass[a4paper,12pt]{article}
\usepackage[left=2.5cm,right=2.5cm,top=2.5cm,bottom=2.5cm]{geometry} % Adjust page margins
\usepackage{xcolor,graphicx,framed}
\usepackage[normalem]{ulem}
\usepackage{amsmath}
\usepackage{gensymb}
\usepackage{chemmacros}
%\usepackage{lastpage} % Required to print the total number of pages

\begin{document}

\newcommand{\HRule}{\rule{\linewidth}{0.4mm}} % Defines a new command for the horizontal lines, change thickness here

%----------------------------------------------------------------------------------------
%	HEADING SECTIONS
%----------------------------------------------------------------------------------------

\begin{minipage}{0.7\textwidth}
\begin{flushleft} 
\textsc{Universidad del Valle de Guatemala \\
Campus Central \\
Facultad de Ciencias y Humanidades \\
Departamento de Qu\'imica \\
Segundo ciclo, 2014 \\
Fisicoqu\'imica 1 \\
}
\end{flushleft}
\end{minipage}
~
\begin{minipage}{0.2\textwidth}
\begin{flushright}
\includegraphics[scale=0.3]{Logo_UVG} % Include a department/university logo
\end{flushright}
\end{minipage}\\

%----------------------------------------------------------------------------------------
%	TITLE SECTION
%----------------------------------------------------------------------------------------

\begin{center}
\HRule \\[0.4cm]
{ \bfseries Ejercicios en clase, 7}\\ % Title of your document
\HRule \\[0.4cm]
\end{center}

%----------------------------------------------------------------------------------------

\begin{enumerate}

 \item \textit{(McQuarrie 20-13)} Calcular el cambio de entrop\'ia de $2.00\;\mbox{mol}$ de $\mbox{H}_2\mbox{O(l)}$ (tomar $\bar{C}_P=75.2\;\mbox{J}\cdot\mbox{K}^{-1}\cdot\mbox{mol}^{-1}$, independiente de la temperatura) cuando se calienta de $10\celsius$ a $90\celsius$ a presi\'on constante. % Problema 6-13.

 \item \textit{(McQuarrie 20-26)} Calcular el cambio de entrop\'ia del sistema, de los alrededores y el cambio total de entrop\'ia cuando un mol de un gas ideal se expande isot\'ermicamente en contra del vac\'io, con una presi\'on interna inicial de $10.0\;\mbox{bar}$ hasta una presi\'on final de $2.00\;\mbox{bar}$ a $300\;\mbox{K}$. % Problema 6-26

 \item Vaporizaci\'on en el punto normal de ebullici\'on ($T_{vap}$) de una sustancia (el punto de ebullici\'on a una atm\'osfera de presi\'on) puede ser considerado como un proceso reversible dado que si la temperatura es disminuida infinitesimalmente debajo de $T_{vap}$ todo el vapor se condensar\'a a l\'iquido, mientras que si se aumenta infinitesimalmente encima de $T_{vap}$ todo el l\'iquido se evaporar\'a. Lo mismo aplica para la fusi\'on en el punto normal de fusi\'on ($T_{fus}$). 
\begin{enumerate}
 \item \textit{(McQuarrie 20-18)} Calcular el cambio de entrop\'ia cuando dos moles de agua se evaporan a $100.0\celsius$. El valor de $\Delta_{vap}\bar{H}$ es $40.65\;\mbox{kJ}\cdot\mbox{mol}^{-1}$. Comentar sobre el signo de $\Delta_{vap}S$. % Problema 6-18

 \item \textit{(McQuarrie 20-19)} Calcular el cambio de entrop\'ia cuando dos moles de agua se funden a $0\celsius$. El valor de $\Delta_{fus}\bar{H}$ es $6.01\;\mbox{kJ}\cdot\mbox{mol}^{-1}$. Comparar con la respuesta obtenida en el inciso anterior. ?`Por qu\'e es $\Delta_{vap}S$ mucho m\'as grande que $\Delta_{fus}S$? % Problema 6-19
\end{enumerate}

 \item \textit{(McQuarrie 20-27)} La capacidad calor\'ifica molar del 1-buteno puede ser expresada como
$$\bar{C}_P(T)/R=0.05641+(0.04635\;\mbox{K}^{-1})T-(2.392\times 10^{-5}\;\mbox{K}^{-2})T^2+(4.80\times 10^{-9}\;\mbox{K}^{-3})T^3$$
en el rango de temperatura $300\;\mbox{K}<T<1500\;\mbox{K}$. Calcular el cambio de entrop\'ia cuando un mol de 1-buteno es calentado de $300\;\mbox{K}$ a $1000\;\mbox{K}$ a presi\'on constante. % Problema 6-27

\end{enumerate}
 
\end{document}
