\documentclass[a4paper,12pt]{article}
\usepackage[left=2.5cm,right=2.5cm,top=2.5cm,bottom=2.5cm]{geometry} % Adjust page margins
\usepackage{xcolor,graphicx,framed}
\usepackage[normalem]{ulem}
\usepackage{amsmath}
\usepackage{gensymb}
\usepackage{chemmacros}
%\usepackage{lastpage} % Required to print the total number of pages

\begin{document}

\newcommand{\HRule}{\rule{\linewidth}{0.4mm}} % Defines a new command for the horizontal lines, change thickness here

%----------------------------------------------------------------------------------------
%	HEADING SECTIONS
%----------------------------------------------------------------------------------------

\begin{minipage}{0.7\textwidth}
\begin{flushleft} 
\textsc{Universidad del Valle de Guatemala \\
Campus Central \\
Facultad de Ciencias y Humanidades \\
Departamento de Qu\'imica \\
Segundo ciclo, 2014 \\
Fisicoqu\'imica 1 \\
}
\end{flushleft}
\end{minipage}
~
\begin{minipage}{0.2\textwidth}
\begin{flushright}
\includegraphics[scale=0.3]{Logo_UVG} % Include a department/university logo
\end{flushright}
\end{minipage}\\

%----------------------------------------------------------------------------------------
%	TITLE SECTION
%----------------------------------------------------------------------------------------

\begin{center}
\HRule \\[0.4cm]
{ \bfseries Ejercicios en clase, 8}\\ % Title of your document
\HRule \\[0.4cm]
\end{center}

%----------------------------------------------------------------------------------------

\begin{enumerate}

 \item 
 \begin{enumerate}
  \item \textit{(McQuarrie 20-29)} Calcular la entrop\'ia de mezcla cuando dos moles de $\mbox{N}_2\mbox{(g)}$ se mezclan con un mol de $\mbox{O}_2\mbox{(g)}$ a la misma temperatura y presi\'on. Asumir comportamiento ideal. % Problema 6-29

  \item \textit{(McQuarrie 20-30)} Mostrar que $\Delta_{mix}\bar{S}=R\ln 2$ si vol\'umenes iguales de cualesquiera dos gases ideales bajo las mismas condiciones son mezclados. % Problema 6-30
 \end{enumerate}

 \item \textit{(McQuarrie 21-45)} Tomando $S^\standardstate[\mbox{CH}_3\mbox{OH(l)}]=126.8\;\mbox{J}\cdot\mbox{K}^{-1}\cdot\mbox{mol}^{-1}$ a $298.15\;\mbox{K}$ y dado que $T_{vap}=337.7\;\mbox{K}$, $\Delta_{vap}\bar{H}(T_{vap})=36.5\;\mbox{kJ}\cdot\mbox{mol}^{-1}$, $\bar{C}_P[\mbox{CH}_3\mbox{OH(l)}]=81.12\;\mbox{J}\cdot\mbox{K}^{-1}\cdot\mbox{mol}^{-1}$ y $\bar{C}_P[\mbox{CH}_3\mbox{OH(g)}]=43.8\;\mbox{J}\cdot\mbox{K}^{-1}\cdot\mbox{mol}^{-1}$, calcular el valor de $S^\standardstate[\mbox{CH}_3\mbox{OH(g)}]$ a $298.15\;\mbox{K}$ y compararlo con el valor experimental de $239.8\;\mbox{J}\cdot\mbox{K}^{-1}\cdot\mbox{mol}^{-1}$. % Problema 7-45

 \item \textit{(McQuarrie 21-42)} Arreglar las siguientes reacciones en orden ascendente de sus valores de $\Delta_rS^\standardstate$ (sin consultar ninguna referencia de valores).
 \begin{enumerate}
  \item $\mbox{S(s)}+\mbox{O}_2\mbox{(g)}\;\rightarrow\;\mbox{SO}_2\mbox{(g)}$
  \item $\mbox{H}_2\mbox{(g)}+\mbox{O}_2\mbox{(g)}\;\rightarrow\;\mbox{H}_2\mbox{O}_2\mbox{(l)}$
  \item $\mbox{CO(g)}+3\,\mbox{H}_2\mbox{(g)}\;\rightarrow\;\mbox{CH}_4\mbox{(g)}+\mbox{H}_2\mbox{O(l)}$
  \item $\mbox{C(s)}+\mbox{H}_2\mbox{O(g)}\;\rightarrow\;\mbox{CO(g)}+\mbox{H}_2\mbox{(g)}$
 \end{enumerate} % Problema 7-42

 \item \textit{(McQuarrie 22-1)} La entalp\'ia molar de vaporizaci\'on del benceno en su punto normal de ebullici\'on ($80.09\celsius$) es $30.72\;\mbox{kJ}\cdot\mbox{mol}^{-1}$. Asumiendo que $\Delta_{vap}\bar{H}$ y $\Delta_{vap}\bar{S}$ se mantienen constantes en sus valores a $80.09\celsius$, calcular el valor de $\Delta_{vap}\bar{G}$ a $75.0\celsius$, $80.09\celsius$ y $85.0\celsius$. Interpretar los resultados f\'isicamente. % Problema 8-1

 \item \textit{(McQuarrie 21-47)} Usar datos de una tabla de entrop\'ias est\'andar para calcular el valor de $\Delta_rS^\standardstate$ de las siguientes reacciones a $25\celsius$ y un bar.
 \begin{enumerate}
  \item $\mbox{C(s, grafito)}+\mbox{O}_2\mbox{(g)}\;\rightarrow\;\mbox{CO}_2\mbox{(g)}$
  \item $\mbox{CH}_4\mbox{(g)}+2\,\mbox{O}_2\mbox{(g)}\;\rightarrow\;\mbox{CO}_2\mbox{(g)}+2\,\mbox{H}_2\mbox{O(l)}$
  \item $\mbox{C}_2\mbox{H}_2\mbox{(g)}+\mbox{H}_2\mbox{(g)}\;\rightarrow\;\mbox{C}_2\mbox{H}_4\mbox{(g)}$
 \end{enumerate} % Problema 7-47

\end{enumerate}
 
\end{document}
