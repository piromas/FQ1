\documentclass[a4paper,12pt]{article}
\usepackage[left=2.5cm,right=2.5cm,top=2.5cm,bottom=2.5cm]{geometry} % Adjust page margins
\usepackage{xcolor,graphicx,framed}
\usepackage[normalem]{ulem}
\usepackage{amsmath}
\usepackage{cases}
\usepackage{gensymb}
\usepackage{chemmacros}
\setlength{\extrarowheight}{0.4cm}

\begin{document}

\newcommand{\HRule}{\rule{\linewidth}{0.4mm}} % Defines a new command for the horizontal lines, change thickness here

%----------------------------------------------------------------------------------------
%	HEADING SECTIONS
%----------------------------------------------------------------------------------------

\begin{minipage}{0.7\textwidth}
\begin{flushleft} 
\textsc{Universidad del Valle de Guatemala \\
Campus Central \\
Facultad de Ciencias y Humanidades \\
Departamento de Qu\'imica \\
Segundo ciclo, 2014 \\
Fisicoqu\'imica 1 \\
}
\end{flushleft}
\end{minipage}
~
\begin{minipage}{0.2\textwidth}
\begin{flushright}
\includegraphics[scale=0.3]{Logo_UVG} % Include a department/university logo
\end{flushright}
\end{minipage}\\

%----------------------------------------------------------------------------------------
%	TITLE SECTION
%----------------------------------------------------------------------------------------

\begin{center}
\HRule \\[0.4cm]
{ \bfseries Soluciones propuestas a los ejercicios en clase, 13}\\ % Title of your document
\HRule \\[0.4cm]
\end{center}

%----------------------------------------------------------------------------------------

\begin{enumerate}

 \item \textbf{\textit{(Chang 7.48)} Haga comentarios sobre cu\'al de los siguientes enunciados es verdadero o falso, y explique brevemente su respuesta:}
\begin{enumerate}
 \item \textbf{Si un componente de una soluci\'on obedece la ley de Raoult, entonces el otro componente tambi\'en debe obedecerla.}



 \item \textbf{En las soluciones ideales, las fuerzas intermoleculares son peque\~nas.}



 \item \textbf{Si se mezclan $15.0\;\mbox{mL}$ de una soluci\'on acuosa de etanol de $3.0\;\mbox{M}$ con $55.0\;\mbox{mL}$ de una soluci\'on acuosa de etanol de $3.0\;\mbox{M}$, el volumen total es de $70.0\;\mbox{mL}$.}



\end{enumerate} % Problema 7.48 de Chang

 \item \textbf{\textit{(Chang 7.36)} Una mezcla de los l\'iquidos $A$ y $B$ muestra un comportamiento ideal. A $84\,\celsius$ la presi\'on total de vapor de una soluci\'on que contiene $1.2\;\mbox{mol}$ de $A$ y $2.3\;\mbox{mol}$ de $B$ llega a $331\;\mbox{mmHg}$. Al agregar otro $mol$ de $B$ a la soluci\'on, la presi\'on de vapor aumenta a $347\;\mbox{mmHg}$. Calcule las presiones de vapor de $A$ y $B$ puros a $84\,\celsius$.} % Problema 7.36 de Chang



 \item \textbf{\textit{(McQuarrie 24-22)} Las presiones de vapor del benceno y tolueno entre $80\,\celsius$ y $110\,\celsius$ en funci\'on de la temperatura en Kelvin est\'an dadas por las f\'ormulas emp\'iricas:}
$$\ln(P_{benceno}^\star/\mbox{torr})=-\frac{3856.6\;\mbox{K}}{T}+17.551$$
\textbf{y}
$$\ln(P_{tolueno}^\star/\mbox{torr})=-\frac{4514.6\;\mbox{K}}{T}+18.397$$
\textbf{Asumiendo que el benceno y el tolueno forman una soluci\'on ideal, usar estas f\'ormulas para construir el diagrama de temperatura-composici\'on de este sistema a una presi\'on ambiente de $760\;\mbox{torr}$.} % Problema 10-22



 \item \textbf{\textit{(McQuarrie 24-48)} Una mezcla de triclorometano y acetona con $x_{acetona}=0.713$ tiene un presi\'on de vapor total de $220.5\;\mbox{torr}$ a $28.2\,\celsius$ y la fracci\'on molar de la acetona en el vapor es $y_{acetona}=0.818$. Dado que la presi\'on de vapor del triclorometano puro a $28.2\,\celsius$ es $221.8\;\mbox{torr}$, calcular la actividad y el coeficiente de actividad (basado en un estado est\'andar seg\'un la ley de Raoult) del triclorometano en la mezcla. Asumir comportamiento ideal para el vapor.}  % Problema 10-48



\end{enumerate}
 
\end{document}
