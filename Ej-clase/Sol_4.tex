\documentclass[a4paper,12pt]{article}
\usepackage[left=2.5cm,right=2.5cm,top=2.5cm,bottom=2.5cm]{geometry} % Adjust page margins
\usepackage{xcolor,graphicx,framed}
\usepackage[normalem]{ulem}
\usepackage{amsmath}
\usepackage{gensymb}
%\usepackage{lastpage} % Required to print the total number of pages

\begin{document}

\newcommand{\HRule}{\rule{\linewidth}{0.4mm}} % Defines a new command for the horizontal lines, change thickness here

%----------------------------------------------------------------------------------------
%	HEADING SECTIONS
%----------------------------------------------------------------------------------------

\begin{minipage}{0.7\textwidth}
\begin{flushleft} 
\textsc{Universidad del Valle de Guatemala \\
Campus Central \\
Facultad de Ciencias y Humanidades \\
Departamento de Qu\'imica \\
Segundo ciclo, 2014 \\
Fisicoqu\'imica 1 \\
}
\end{flushleft}
\end{minipage}
~
\begin{minipage}{0.2\textwidth}
\begin{flushright}
\includegraphics[scale=0.3]{Logo_UVG} % Include a department/university logo
\end{flushright}
\end{minipage}\\

%----------------------------------------------------------------------------------------
%	TITLE SECTION
%----------------------------------------------------------------------------------------

\begin{center}
\HRule \\[0.4cm]
{ \bfseries Soluciones propuestas a los ejercicios en clase, 4}\\ % Title of your document
\HRule \\[0.4cm]
\end{center}

%----------------------------------------------------------------------------------------

\begin{enumerate}

 \item \textbf{\textit{(McQuarrie 19-25)} Una muestra de $25.0\;\mbox{g}$ de cobre a $363\;\mbox{K}$ es introducida en $100.0\;\mbox{g}$ de agua a $293\;\mbox{K}$. El cobre y el agua r\'apidamente llegan a la misma temperatura por el proceso de tranferencia de calor del cobre al agua. Calcular la temperatura final del agua. La capacidad calor\'ifica molar del cobre es $24.5\;\mbox{J}\cdot\mbox{K}^{-1}\cdot\mbox{mol}^{-1}$.} % Problema 5-25

Como se habla de una transferencia de calor del cobre al agua, no hay transferencia de calor hacia los alrededores (proceso adiab\'atico si se considera como sistema al cobre y agua): 
$$q=q_{Cu}+q_{H_2O}=0$$
Sabemos que $n_{Cu}=25.0\;\mbox{g}\left(\frac{1\;\mbox{mol}}{63.546\;\mbox{g}}\right)=0.39\;\mbox{mol}$ y que: 
$$C_{Cu}=\bar{C}_{Cu}\cdot n_{Cu}=(24.5\;\mbox{J}\cdot\mbox{K}^{-1}\cdot\mbox{mol}^{-1})(0.39\;\mbox{mol})=9.64\;\mbox{J}\cdot\mbox{K}^{-1}$$
Entonces:
$$q_{Cu}=-q_{H_2O}\;\rightarrow\; C_{Cu}\Delta T_{Cu}=-m_{H_2O}c_{H_2O}\Delta T_{H_2O}$$
$$\rightarrow\; (9.64\;\mbox{J}\cdot\mbox{K}^{-1})(T_f-363\;\mbox{K})=-(100.0\;\mbox{g})(4.184\;\mbox{J}\cdot\mbox{g}^{-1}\cdot\mbox{K}^{-1})(T_f-293\;\mbox{K})$$
$$\rightarrow\;(9.64\;\mbox{J}\cdot\mbox{K}^{-1}+418.4\;\mbox{J}\cdot\mbox{K}^{-1})T_f=(9.64\;\mbox{J}\cdot\mbox{K}^{-1})(363\;\mbox{K})+(418.4\;\mbox{J}\cdot\mbox{K}^{-1})(293\;\mbox{K})$$
Lo que da como temperatura final del agua: $T_f=295\;\mbox{K}$.

 \item \textbf{\textit{(McQuarrie 19-20)} Cierta cantidad de $N_2(g)$ a $298\;\mbox{K}$ se comprime reversible y adiab\'aticamente de $20.0\;\mbox{dm}^3$ a $5.00\;\mbox{dm}^3$. Asumiendo un comportamiento ideal, calcular la temperatura final del $N_2(g)$. Tomar $\bar{C}_V=5R/2$.} % Problema 5-20

Como es un proceso adiab\'atico $q=0$ y $dU=dw$; siendo un proceso reversible, podemos calcular el trabajo usando la presi\'on interna, dada por la ecuaci\'on del gas ideal, $P=nRT/V$, as\'i que:
$$dU=dw\;\rightarrow\;C_VdT=-\frac{nRT}{V}dV$$
$$\rightarrow\;\int_{T_1}^{T_2}\frac{\bar{C}_V}{RT}dT=-\int_{V_1}^{V_2}\frac{dV}{V}\;\rightarrow\;\frac{5}{2}\ln\frac{T_2}{T_1}=-\ln\frac{V_2}{V_1}$$
$$\rightarrow\;\ln\frac{T_2}{T_1}=\ln\left(\frac{V_1}{V_2}\right)^{2/5}\;\rightarrow\; T_2=T_1\left(\frac{V_1}{V_2}\right)^{2/5}$$
Por lo que:
$$T_2=(298\;\mbox{K})\left(\frac{20.0\;\mbox{dm}^3}{5.00\;\mbox{dm}^3}\right)^{2/5}=519\;\mbox{K}$$

 \item \textbf{\textit{(McQuarrie 19-23)} El valor de $\Delta H$ a $25\celsius$ y un $bar$ es $+290.9\;\mbox{kJ}$ para la reacci\'on
$$2\,ZnO(s)+2\,S(s)\rightarrow 2\,ZnS(s)+O_2(g)$$
Asumiendo un comportamiento ideal, calcular el valor de $\Delta U$ para esta reacci\'on.} % Problema 5-23

La relaci\'on entre la energ\'ia interna y la entalp\'ia a temperatura constante ($T=25\celsius+273=298\;\mbox{K}$), asumiendo comportamiento ideal, es:
$$\Delta H=\Delta U+\Delta(PV)=\Delta U+\Delta n_{g}RT$$
Recordando que el valor de $\Delta(PV)$ se puede calcular con el cambio de moles de gases entre productos y reactivos, que en este caso tiene el valor de $\Delta n_{g}=1$, por lo que:
$$\Delta U=\Delta H-\Delta n_{g}RT=290.9\;\mbox{kJ}-(1\;\mbox{mol})(8.314\;\mbox{J}\cdot\mbox{mol}^{-1}\cdot\mbox{K}^{-1})(298\;\mbox{K})=288.4\;\mbox{kJ}$$

 \item \textbf{\textit{(McQuarrie 23-12)} Un mol de un gas ideal monoat\'omico que est\'a inicialmente a una presi\'on de $2.00\;\mbox{bar}$ y una temperatura de $273\;\mbox{K}$ es llevado a una presi\'on final de $4.00\;\mbox{bar}$ por un camino reversible definido por $P/V=\mbox{constante}$. Calcular los valores de $\Delta U$, $\Delta H$, $q$ y $w$ para este proceso. Tomar $\bar{C}_V=12.5\;\mbox{J}\cdot\mbox{mol}^{-1}\cdot\mbox{K}^{-1}$.} % Problema 5-12

Como tenemos un gas ideal, podemos determinar el volumen inicial con los datos iniciales:
$$V_1=\frac{nRT_1}{P_1}=\frac{(1\;\mbox{mol})(0.08314\;\mbox{L}\cdot\mbox{bar}\cdot\mbox{K}^{-1}\cdot\mbox{mol}^{-1})(273\;\mbox{K})}{2.00\;\mbox{bar}}=11.35\;\mbox{L}$$
Adem\'as, tenemos que $P/V=\mbox{constante}$, as\'i que:
$$\frac{P_1}{V_1}=\frac{2.00\;\mbox{bar}}{11.35\;\mbox{L}}=0.176\;\mbox{bar}\cdot\mbox{L}^{-1}=\frac{P_2}{V_2}$$
$$\rightarrow\;V_2=\frac{P_2}{0.176\;\mbox{bar}\cdot\mbox{L}^{-1}}=\frac{4.00\;\mbox{bar}}{0.176\;\mbox{bar}\cdot\mbox{L}^{-1}}=22.70\;\mbox{L}$$
Con lo cual podemos determinar la temperatura final, como es un gas ideal:
$$T_2=\frac{P_2V_2}{nR}=\frac{(4.00\;\mbox{bar})(22.70\;\mbox{L})}{(1\;\mbox{mol})(0.08314\;\mbox{L}\cdot\mbox{bar}\cdot\mbox{K}^{-1}\cdot\mbox{mol}^{-1})}=1092\;\mbox{K}$$
Con estos datos podemos calcular el cambio de energ\'ia interna, usando que $C_V$ es constante:
$$C_V=\bar{C}_V\cdot n=(12.5\;\mbox{J}\cdot\mbox{mol}^{-1}\cdot\mbox{K}^{-1})(1\;\mbox{mol})=12.5\;\mbox{J}\cdot\mbox{K}^{-1}$$
Por lo que:
$$\Delta U=C_V\Delta T=(12.5\;\mbox{J}\cdot\mbox{K}^{-1})(1092\;\mbox{K}-273\;\mbox{K})=10.2\;\mbox{kJ}$$
Como es un proceso reversible con $P/V=0.176\;\mbox{bar}\cdot\mbox{L}^{-1}$, podemos calcular el trabajo:
$$w_{rev}=-\int P_{int}dV=-0.176\;\mbox{bar}\cdot\mbox{L}^{-1}\int_{V_1}^{V_2}VdV=-\frac{0.176\;\mbox{bar}\cdot\mbox{L}^{-1}}{2}(V_2^2-V_1^2)$$
$$\rightarrow\;w=-\frac{0.176\;\mbox{bar}\cdot\mbox{L}^{-1}}{2}((22.70\;\mbox{L})^2-(11.35\;\mbox{L})^2)\left(\frac{8.314\;\mbox{J}}{0.08314\;\mbox{L}\cdot\mbox{bar}}\right)=-3.40\;\mbox{kJ}$$
Con lo cual podemos determinar el calor:
$$\Delta U=q+w\;\rightarrow\;q=\Delta U-w=10.2\;\mbox{kJ}-(-3.40\;\mbox{kJ})=13.6\;\mbox{kJ}$$
Por \'ultimo, podemos calcular la entalp\'ia de la siguiente manera:
\begin{align*}
\Delta H &=\Delta U+\Delta(PV)=10.2\;\mbox{kJ}+\left((4.00\;\mbox{bar})(22.70\;\mbox{L})-(2.00\;\mbox{bar})(11.35\;\mbox{L})\right) \\
&=10.2\;\mbox{kJ}+68.1\;\mbox{L}\cdot\mbox{bar}\left(\frac{8.314\;\mbox{J}}{0.08314\;\mbox{L}\cdot\mbox{bar}}\right)=10.2\;\mbox{kJ}+6.8\;\mbox{kJ}=17.0\;\mbox{kJ}
\end{align*}

 \item \textbf{\textit{(McQuarrie 19-33)} Derivar $H=U+PV$ con respecto a $V$ a temperatura constante para mostrar que $(\partial H/\partial V)_T=0$ para un gas ideal (asumir que para un gas ideal, $U$ solo depende de $T$).} % Problema 5-33

Sacando la derivada parcial, usando la regla del producto para el segundo t\'ermino:
$$\left(\frac{\partial H}{\partial V}\right)_{T}=\left(\frac{\partial }{\partial V}\right)_{T}(U+PV)=\left(\frac{\partial U}{\partial V}\right)_{T}+\left(\frac{\partial P}{\partial V}\right)_{T}V+P\left(\frac{\partial V}{\partial V}\right)_{T}$$
$$\rightarrow\;\left(\frac{\partial H}{\partial V}\right)_{T}=\left(\frac{\partial U}{\partial V}\right)_{T}+V\left(\frac{\partial }{\partial V}\right)_{T}\left(\frac{nRT}{V}\right)+P=\left(\frac{\partial U}{\partial V}\right)_{T}-V\frac{nRT}{V^2}+\frac{nRT}{V}$$
Como para un gas ideal $U$ solo depende de $T$, $\left(\frac{\partial U}{\partial V}\right)_{T}=0$, por lo que:
$$\left(\frac{\partial H}{\partial V}\right)_{T}=\left(\frac{\partial U}{\partial V}\right)_{T}=0$$

\end{enumerate}

\end{document}
