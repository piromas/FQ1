\documentclass[a4paper,12pt]{article}
\usepackage[left=2.5cm,right=2.5cm,top=2.5cm,bottom=2.5cm]{geometry} % Adjust page margins
\usepackage{xcolor,graphicx,framed}
\usepackage[normalem]{ulem}
\usepackage{amsmath}
\usepackage{cases}
\usepackage{gensymb}
\usepackage{chemmacros}
\setlength{\extrarowheight}{0.4cm}

\begin{document}

\newcommand{\HRule}{\rule{\linewidth}{0.4mm}} % Defines a new command for the horizontal lines, change thickness here

%----------------------------------------------------------------------------------------
%	HEADING SECTIONS
%----------------------------------------------------------------------------------------

\begin{minipage}{0.7\textwidth}
\begin{flushleft} 
\textsc{Universidad del Valle de Guatemala \\
Campus Central \\
Facultad de Ciencias y Humanidades \\
Departamento de Qu\'imica \\
Segundo ciclo, 2014 \\
Fisicoqu\'imica 1 \\
}
\end{flushleft}
\end{minipage}
~
\begin{minipage}{0.2\textwidth}
\begin{flushright}
\includegraphics[scale=0.3]{Logo_UVG} % Include a department/university logo
\end{flushright}
\end{minipage}\\

%----------------------------------------------------------------------------------------
%	TITLE SECTION
%----------------------------------------------------------------------------------------

\begin{center}
\HRule \\[0.4cm]
{ \bfseries Gu\'ia para la evaluaci\'on 4}\\ % Title of your document
\HRule \\[0.4cm]
\end{center}

%----------------------------------------------------------------------------------------

\section*{Cap\'itulo 5, Mezclas simples}

\subsection*{Propiedades termodin\'amicas de mezcla}

\begin{itemize}
 \item ?`Qu\'e son las propiedades molares parciales y c\'omo se calculan? Por ejemplo, el volumen parcial molar: $V_J=\left(\frac{\partial V}{\partial n_J}\right)_{P,T,n_i,i\neq J}$.
 \item ?`C\'omo se determinar una propiedad extensiva de una mezclas? Por ejemplo, la masa total: $m=M_An_A+M_Bn_B$.
 \item ?`C\'omo se interpreta la ecuaci\'on de Gibbs-Duhem y qu\'e relaciona? $\sum_J n_Jd\mu_J=0$.
 \item C\'omo se calculan y c\'omo interpretar las siguientes cantidades para gases ideales o soluciones ideales:
 \begin{itemize}
  \item Energ\'ia de Gibbs de mezcla: $\Delta_{mix}G=nRT(x_A\ln x_A+x_b\ln x_B)$.
  \item Entrop\'ia de mezcla: $\Delta_{mix}S=-\left(\frac{\partial\Delta_{mix}G}{\partial T}\right)_{P,n_A,n_B}=-nR(x_A\ln x_A+x_B\ln x_B)$.
  \item Entalp\'ia de mezcla: $\Delta_{mix}H=\Delta_{mix}G+T\Delta_{mix}G=0$.
  \item Cambio de volumen de mezcla: $\Delta_{mix}V=\left(\frac{\partial\Delta_{mix}G}{\partial P}\right)_{T,n_A,n_B}=0$.
 \end{itemize}
\end{itemize}

\subsection*{Potencial qu\'imico}

\begin{itemize}
 \item Definici\'on e interpretaci\'on: $\mu_J=\left(\frac{\partial G}{\partial n_J}\right)_{P,T,n_i,i\neq J}$.
 \item Ecuaci\'on fundamental de la termodin\'amica qu\'imica: $dG=VdP-SdT+\sum_i \mu_idn_i$.
 \item Condici\'on de equilibrio en t\'erminos del potencial qu\'imico: en el equilibrio el potencial qu\'imico de una sustancia es el mismo en toda la muestra, no importando que se encuentre presente en distintas fases, $\mu_A(\alpha)=\mu_A(\beta)$.
 \item ?`Qu\'e es la actividad de una sustancia y cu\'ales son los estados est\'andar?
 \item C\'omo es el potencial qu\'imico y en t\'erminos de qu\'e se representa de los siguientes casos:
 \begin{itemize}
  \item Gases ideales: $\mu_A=\mu_A^\standardstate+RT\ln\frac{P_A}{P^\standardstate}$.
  \item Gases reales: $\mu_A=\mu_A^\standardstate+RT\ln\frac{f_A}{P^\standardstate}$.
  \item Soluciones ideales: $\mu_A=\mu_A^\star+RT\ln x_A$.
  \item Solutos en soluciones ideales diluidas: $\mu_B=\mu_B^\standardstate+RT\ln x_B$, con $\mu_B^\standardstate=\mu_B^\star+RT\ln\frac{K_B}{P_B^\star}$.
  \item Soluciones reales: $\mu_A=\mu_A^\star+RT\ln a_A=\mu_A^\star+RT\ln x_A+RT\ln\gamma_A$.
  \item Solutos reales: $\mu_B=\mu_B^\standardstate+RT\ln a_B=\mu_B^\standardstate+RT\ln x_B+RT\ln\gamma_B$.
 \end{itemize}
\end{itemize}

\subsection*{Soluciones ideales}

\begin{itemize}
 \item ?`Qu\'e es la presi\'on de vapor, $P_A$?
 \item Definici\'on de la ley de Dalton y c\'omo se usa: $P=\sum_iP_i$ con $P_i=x_iP$.
 \item Definici\'on de la ley de Raoult y c\'omo se usa: $P_A=x_AP_A^\star$. ?`Cu\'al es su relaci\'on con las soluciones ideales? En las soluciones ideales, ambos componentes siguen la ley de Raoult.
 \item Definici\'on de la ley de Henry y c\'omo se usa: $P_B=x_BK_B'$ \'o $P_B=m_BK_B$. ?`Cu\'al es su relaci\'on con las soluciones ideales diluidas? En las soluciones ideales diluidas el solvente sigue la ley de Raoult y el soluto la ley de Henry.
\end{itemize}

\subsection*{Soluciones no ideales}

\begin{itemize}
 \item ?`Qu\'e tipo de desviaciones a la ley de Raoult hay y c\'omo se interpretan con respecto a las fuerzas intermoleculares? Desviaci\'on positiva (fuerzas repulsivas) y desviaci\'on negativa (fuerzas atractivas).
 \item ?`Qu\'e son los aze\'otropos y c\'omo se relacionan con las desviaciones de la ley de Raoult? Una mezcla azeotr\'opica o aze\'otropo es una soluci\'on que cuando ebulle a una temperatura espec\'ifica la composici\'on del vapor es la misma que la composici\'on de la mezcla l\'iquida. Para desviaciones positivas de la ley de Raoult se alcanza una mezcla azeotr\'opica (para una composici\'on espec\'ifica) de punto de ebullici\'on m\'inima, mientras que para desviaciones negativas de la ley de Raoult se forma una mezcla azeotr\'opica (para una composici\'on espec\'ifica) de punto de ebullici\'on m\'aximo (con respecto a los puntos de ebullici\'on de los solventes puros).
\end{itemize}

\subsection*{Equilibrio de fases en sistemas binarios}

\begin{itemize} 
 \item Uso de la regla de las fases en sistemas binarios: $F=C-P+2$, con $C=c-r-a$.
 \item Fracciones molares: en la fase l\'iquida, $x_A=\frac{n_A(l)}{n_T(l)}$; en la fase de vapor, $y_A=\frac{n_A(v)}{n_T(v)}$; en t\'erminos de ambas fases, $z_A=\frac{n_A}{n_T}=\frac{n_A(l)+n_A(v)}{n_T(l)+n_T(v)}$.
 \item Presi\'on total en t\'erminos de la composici\'on de la mezcla l\'iquida: $P=P_B^\star+(P_A^\star-P_B^\star)x_A$.
 \item Presi\'on total en t\'erminos de la composici\'on de la mezcla de vapor: $P=\frac{P_A^\star P_B^\star}{P_A^\star+(P_B^\star-P_A^\star)y_A}$.
 \item Diagramas de presión-composici\'on, $P$ vrs $z_A$, y diagramas de temperatura-composici\'on, $T$ vrs $z_A$. 
 \begin{itemize}
  \item ?`C\'omo son los diagramas?
  \item ?`Qu\'e indican las lineas en cada diagrama?
  \item ?`Qu\'e indican las regiones en cada diagrama?
  \item ?`C\'omo determinar las curvas de la presi\'on de vapor total en t\'erminos de la composici\'on de la mezcla l\'iquida o de la mezcla de vapor?
  \item A una temperatura o presi\'on dada, ?`c\'omo determinar la composici\'on en las fases?
  \item Regla de la palanca y su uso para determinar la proporci\'on entre las fases: $\frac{n_\alpha}{n_\beta}=\frac{l_\beta}{n_\alpha}$.
  \item ?`C\'omo entender la destilaci\'n seg\'un el diagrama de temperatura-composici\'on?
  \item ?`C\'omo determinar el n\'umero de platos en una destilaci\'on fraccionada a partir del diagrama de temperatura-composici\'on?
 \end{itemize}

 \item ?`C\'omo son los diagramas de temperatura-composici\'on entre l\'iquidos parcialmente miscibles?
 \item ?`Qu\'e es la temperatura cr\'itica de disoluci\'on superior, $T_{uc}$, y la temperatura cr\'itica de disoluci\'on inferior, $T_{lc}$?
 \item ?`C\'omo son los diagramas de temperatura-composici\'on entre  l\'iquidos y s\'olidos?
 \item ?`Qu\'e es el punto eut\'ectico?
\end{itemize}

\subsection*{Propiedades coligativas}

\begin{itemize}
 \item ?`Qu\'e unidades de concentración hay y qu\'e cantidades relacionan? Porcentaje en peso, $\%(p/p)=\frac{m_{soluto}}{m_{total}}\times 100$; fracci\'on molar, $x_A=\frac{n_A}{n_T}$; molalidad, $m=\frac{n_A}{m_B}$; molaridad, $M=\frac{n_A}{V_T}$.
 \item ?`Cu\'ales son las propiedades coligativas y de qu\'e dependen? Disminuci\'on de presi\'on de vapor, elevaci\'on del punto de ebullici\'on, depresi\'on del punto de fusi\'on y presi\'on osm\'otica; todas dependen de la cantidad de part\'iculas disueltas y no de la identidad o forma de las part\'iculas.
 \item ?`C\'omo explicarlas seg\'un el potencial qu\'imico? El potencial qu\'imico de un l\'iquido est\'a disminuido en funci\'on de $\mu_A=\mu_A^\star+RT\ln x_A$, dado que $x_A\neq 1$ cuando hay un soluto disuelto. En particular, para la depresi\'on del punto de fusi\'on y la elevaci\'on del punto de ebullici\'on, al disminuir el potencial qu\'imico del l\'iquido, en la gr\'afica entre $\mu$ vrs. $T$ para las distintas fases del solvente, la intersecci\'on entre s\'olido y l\'iquido se corre a la izquierda y la intersecci\'on entre l\'iquido y vapor se corre a la derecha.
 \item ?`Qu\'e principio relacionado con el potencial qu\'imico se usa en cada caso? Elevaci\'on del punto de ebullici\'on, $\mu_A^\star(g)=\mu_A^\star(l)+RT\ln x_A$; depresi\'on del punto de fus\'on, $\mu_A^\star(s)=\mu_A^\star(l)+RT\ln x_A$; presi\'on osm\'otica, $\mu_A^\star(P)=\mu_A(x_A,P+\Pi)$.
 \item ?`C\'omo se determinan cuantitativamente cada una? Disminuci\'on de la presi\'on de vapor, $\Delta P=x_BP_A^\star$; elevaci\'on del punto de ebullici\'on, $\Delta T=K_bm$ con $K_b=\frac{RT_b^2M}{\Delta_{vap}H}$; depresi\'on del punto de fusi\'on, $\Delta T=K_fm$ con $K_f=\frac{RT_f^2M}{\Delta_{fus}H}$; presi\'on osm\'otica, $\Pi=\frac{n_B}{V}RT$.
 \item ?`En qu\'e l\'imite se aplican las fórmulas cuantitativas anteriores? Soluciones ideales diluidas en el que el solvente es no-vol\'atil y sin electr\'olitos: en el solvente aplica la ley de Raoult y la solucion es dilu\'ida como para que $x_B\ll 1$ (para que $\ln (1-x_B)\approx -x_B$) y que $\Delta_{vap}H$ de la soluci\'on sea igual a la del solvente puro y constante con respecto al cambio de temperatura (si la soluci\'on es diluida, el cambio de temperatura ser\'a muy peque\~no).
\end{itemize}

\section*{Cap\'itulo 6, Equilibrio qu\'imico}

\subsection*{Reacciones qu\'imicas espont\'aneas}

\begin{itemize}
 \item ?`Qu\'e es el coeficiente estequiom\'etrico, $\nu_i$? ?`Qu\'e signo tiene para los reactivos y qu\'e signo tiene para los productos?
 \item ?`Qu\'e es el grado de avance o extensi\'on de reacci\'on, $\xi=\frac{n_i-n_{i,0}}{\nu_i}$?
 \item Definici\'on de la energ\'ia de Gibbs de reacci\'on y su interpretaci\'on: $\Delta_rG=\left(\frac{\partial G}{\partial\xi}\right)_{P,T}=\sum_i\nu_i\mu_i$.
 \item Condici\'on de equilibrio de una reacci\'on qu\'imica: $\Delta_rG=0$.
 \item ?`C\'omo usar $\Delta_rG$ para determinar si una reacci\'on es espont\'anea o no?
 \item ?`A cu\'ales reacciones se les denomina exerg\'onicas y a cu\'ales enderg\'onicas?
 \item ?`Qu\'e es el cociente de reacci\'on, $Q$, y c\'omo se calcula? $Q=\prod_J a_J^{\nu_J}$.
 \item ?`Qu\'e es la energ\'ia de Gibbs de reacci\'on est\'andar, $\Delta_rG^\standardstate$?
 \item ?`Cu\'al es la relaci\'on entre la energ\'ia de Gibbs de reacci\'on est\'andar y los valores de energ\'ia de Gibbs de formaci\'on? $\Delta_rG^\standardstate=\sum_J\nu_J\cdot\Delta_fG^\standardstate(J)$.
 \item ?`Qu\'e es la constante de equilibrio, $K$, y c\'omo se calcula? $K=\left(\prod_J a_J^{\nu_J}\right)_{equilibrio}$.
 \item Relaci\'on entre la constante de equilibrio con la energ\'ia de Gibbs de reacci\'on est\'andar: $\Delta_rG^\standardstate=-RT\ln K$.
 \item C\'omo se calculan las constantes de equilibrio de los siguientes casos:
 \begin{itemize}
  \item Gases ideales: $K_P=\prod_J (P_J/P^\standardstate)^{\nu_J}$.
  \item Gases reales: $K_f=\prod_J (f_J/P^\standardstate)^{\nu_J}$.
  \item Soluciones ideales: $K_x=\prod_J x_J^{\nu_J}$.
  \item Soluciones ideales (en t\'erminos de molalidades): $K=\prod_J (m_J/m^\standardstate)^{\nu_J}$.
  \item Soluciones ideales (en t\'erminos de molaridades): $K=\prod_J ([J]/c^\standardstate)^{\nu_J}$.
  \item Soluciones reales, $K=\prod_J a_J^{\nu_J}$.
 \end{itemize}
 \item ?`Cu\'ales son las relaciones entre las constantes de equilibrio? $K=K_\gamma K_m$ y $K_P=K_c\times \left(\frac{c^\standardstate RT}{P^\standardstate}\right)^{\Delta\nu}$, donde $\Delta \nu=\sum_J\nu_J$.
\end{itemize}

\subsection*{Respuesta del equilibrio ante cambio de las condiciones}

\begin{itemize}
 \item ?`Cu\'al es el principio de Le Chatelier?
 \item ?`C\'omo afectan los cambios de presi\'on a una reacci\'on en equilibrio?
 \item ?`C\'omo afectan los cambios de temperatura a una reacci\'on en equilibrio?
 \item Definici\'on de la ecuaci\'on de van 't Hoff y c\'omo se usa: $\frac{d\ln K}{dT}=-\frac{\Delta_rH^\standardstate}{RT^2}$, \'o $\ln\frac{K_2}{K_1}=-\frac{\Delta_rH^\standardstate}{R}\left(\frac{1}{T_2}-\frac{1}{T_1}\right)$, \'o $\ln K=-\frac{\Delta_rH}{RT}+\frac{\Delta_rS}{R}$.
\end{itemize}
 
\end{document}
